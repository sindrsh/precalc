\newcommand{\plro}{\rg[Parameteriseringen av et plan i rommet]{
		Et plan $ \alpha $ som med inneholder punktet $ {P=(x_0, y_0, z_0)} $ og to ikke-parallelle vektorer $ {\vec{u}=[a_1, b_1, c_1]} $ og ${ \vec{v}=[a_2, b_2, c_3]} $ kan parameteriseres ved
			\[\alpha: \left\lbrace{
		\begin{array}{lll}
		x= x_0 + a_1s+ a_2t   \\
		y= y_0 + b_1s+ b_2t    \\
		z= z_0 + c_1s+ c_2t 
		\end{array}
	}\right. \]	
	hvor $ s, t \in \mathbb{R}$. 
	}}
\newcommand{\plroe}{\eks[2]{\label{plroe2}
		Et plan $ \alpha $ er gitt ved ligningen
		\[ 3x-y-2z+6=0 \]
		\textbf{a)} Finn en parameterisering til planet.\os
		
		\textbf{b)} Finn et punkt som ligger i planet.
		
		\sv
		\textbf{a)} For å finne en parameterisering for et plan gitt av en ligning, står vi fritt til selv å velge to av $ {x, y} $ og $ z $ som lik hver av parameteriseringsvariablene. Vi velger her $ {x=s} $ og $ {z=t} $, og får at \vs
		\alg{3s-y-2t+6&=0 \\
			y&= 3s+2t-6}
		Parameteriseringen blir da
		\[\alpha: \left\lbrace{
			\begin{array}{lll}
			x=s   \\
			y= -6+ 3s + 2t   \\
			z= t 
			\end{array}
		}\right. \]
		\textbf{b)} Ut ifra parameteriseringen ser vi at et punkt i planet må være $ (0, -6, 0) $
	}}
\newcommand{\liro}{\rg[Linje i rommet]{
		Ei linje \textit{l} som går gjennom punktet $ {A=(x_0, y_0, z_0)} $ og har retningsvektor $ {\vec{r}=[a, b, c] }$ kan parameteriseres ved
		\begin{equation}l: \left\lbrace{
			\begin{array}{lll}
			x= x_0 + a_2t   \\
			y= y_0 + b_2t    \\
			z= z_0 + c_2t 
			\end{array}
		}\right. 
		\end{equation}
	hvor $ t\in\mathbb{R} $.
	}}
\newcommand{\liroe}{\eks{\label{lirep}
		Ei linje går gjennom punktene ${ A=(-2, 2, 1)} $ og $ {B=(2, 4, -5)} $.\os
		
		\textbf{a)} Finn en parameterisering for linja $ l $ som går gjennom $ A $ og $ B $. \os
		\textbf{b)} Sjekk om punktet ${C= (-5, 3, 6)} $ ligger på linja.
		
		\sv
		\textbf{a)}
		Vektoren $ \vv{AB} $ er en retningsvektor for linja:
		\alg{
			\vv{AB}&=[2-(-2),4-2, -5-1] \\
			&= [4, 2, -6] \\
			&= 2[2, 1, -3]
		}
		Vi bruker den forkortede retningsvektoren i kombinasjon med $ A $, og får at
		\[l: \left\lbrace{
			\begin{array}{lll}
			x=-2 + 2t   \\
			y= 2 + t    \\
			z= 1 -3t
			\end{array}
		}\right. \]
		\textbf{b)} Skal $ C $ ligge på $ l $, må parameteriseringen gi oss koordinaten til $ C $ for rett valg av $ t $. Skal for eksempel $ y$-koordinaten bli riktig, må vi ha at
		\alg{
			2+t &= 3 \\
			t &= 1
		} 
	For $ {t=1 }$ blir ${ x=0 }$, men $ x $-koordinaten til $ C $ er $ -5 $, altså ligger ikke $ C $ på linja.
	}}
\newcommand{\plaro}{\rg[Ligningen til et plan i rommet]{
		Et plan med normalvektor $ {n=[a, b, c]} $ kan uttrykkes ved ligningen
		\begin{equation}\label{planlig}
		a(x-x_0) + b(y-y_0) + c(z-z_0)=0
		\end{equation}
		hvor $ A=(x_0, y_0, z_0) $ er et vilkårlig punkt i planet. \vsk\\
		
		Eventuelt kan man skrive
		\begin{equation}\label{planlig2}
		ax + by + zc + d=0
		\end{equation}
		hvor $ -(ax_0 + by_0 +cz_0)=d $.
	}}	
\newcommand{\plaroe}{\eks[1]{\label{plroe1}
		Et plan er utspent av vektorene $ {\vec{u}=[1, -2, 2] }$ og $ \vec{v}=[-3, \\3, 1] $ og inneholder punktet $ {A=(-3, 3, 4)} $. Finn en ligning for planet.
		
		\sv
		En normalvektor til planet kan vi finne ved
		\footnotesize
		\alg{\vec{u}\times\vec{v} &= \begin{vmatrix}
				\vec{e}_1 & \vec{e}_2 & \vec{e}_3 \\
				1 & -2 & 2 \\
				-3 & 3 & 1
			\end{vmatrix} \\
			&= (-2\cdot1-3\cdot2)\vec{e}_1 -(1\cdot1-(-3)\cdot2)\vec{e}_2 + (1\cdot3-(-2)\cdot(-3))\vec{e}_3 \\
			&= [-8, -7, -3] \\
			&= -[8, 7, 3]
		} \normalsize
		Vi har nå en normalvektor og et punkt i planet, og får dermed ligningen
		\alg{
			8(x+3)+7(y-3)+3(z-4) &=0 \\
			8x+24+7y-21+3z-12 &= 0 \\
			8x+7y+3z-9=0
		}\vds
	}}	
\newcommand{\kule}{\rg[Kuleligningen]{
		Ligningen for en kuleflate med radius $ r $ og sentrum $ {S=(x_0, y_0, z_0)} $ er gitt ved
		\begin{equation}
		(x-x_0)^2+(y-y_0)^2+(z-z_0)^2=r^2
		\end{equation}\vds
	}}	
\newcommand{\kulee}{\label{kulee}	\eks[1]{
		En kuleflate er beskrevet ved ligningen
		\[ x^2-6 x+y^2+4 y+z^2-4 z-19=0 \]	
		\textbf{a)} Finn sentrum $ S $ og radiusen til kula. \os
		\textbf{b)} Vis at punktet $A= (7, -6, 4) $ ligger på kuleflaten.\os
		\textbf{c)} Finn tangentplanet til kuleflaten i punktet $ A $.
		
		\sv
		\textbf{a)} For å løse denne oppgaven må vi finne de fullstendige kvadratene:
		\algv{x^2-6x &= (x-3)^2-(-3)^2 \\[5 pt]
			y^2+4y &= (y+2)^2-2^2 \\[5 pt]
			z^2-4z &= (z-2)^2-(-2)^2
		}
		Dermed får vi at
		\algv{
			x^2-6 x+y^2+4 y+z^2-4 z-19&=0	\\
			(x-3)^2 +(y+2)^2 +(z-2)^2-3^2-2^2-2^2-19 &= 0  \\
			(x-3)^2 +(y+2)^2 +(z-2)^2 &= 36 \\
			&= 6^2
		}
		Kula har altså sentrum i punktet $S= (3, -2, 2) $ og radius lik 6.\vsk\\
		
		\textbf{b)} Skal $ A $ ligge på kuleflaten, må koordinatene til $ A $ oppfylle kuleligningen:
		\alg{
		(7-3)^2 +(-6+2)^2 +(4-2)^2 &= 36 \\
		4^2 + (-4)^2 + 2^2 &= 36\\
		36 &= 36
	}
	Dermed har vi vist det vi skulle. \vsk
	
	\textbf{c)} Tangentplanet står normalt på kuleflaten i $ A $ og har derfor $ \vv{SA} $ som normalvektor:
	\alg{
	\vv{SA} &= [7-3, -6-(-2), 4-2] \\
	&= [4, -4, 2] \\
	&=2[2, -2, 1]
	}
	Altså er $ [2, -2, 1] $ en normalvektor for tangentplanet. Av (\ref{planlig}) kan dette planet uttrykkes ved ligningen
	\alg{
	2(x-7)-2(y-(-6))+1(z-4)&=0 \\
	2(x-7)-2(y+6)+1(z-4)&=0\\
	2x-14-2y-12+z-4 &= 0\\
	2x-2y-z-30=0
	}\vds
	}}
\newcommand{\avpli}{\rg[Avstand mellom punkt og linje]{
		Avstanden $ h $ mellom et punkt $ B $ og en linje gitt av punktet $ A $ og retningsvektoren $ \vec{r} $ er gitt som
		\begin{equation}
		h = \frac{\left|\vv{AB}\times \vec r\,\right| }{|\vec{r}|}
		\end{equation}\vs
	}}
\newcommand{\avplp}{\rg[Avstand mellom punkt og plan]{
		Avstanden $ h $ mellom et punkt ${ A=(x_0, y_0, z_0)} $ og et plan beskrevet av ligningen $ {ax + by + cz +d = 0 }$, er gitt som
		\nreq{h= \frac{|a x_0 + b y_0 + c z_0 + d|}{\sqrt{a^2 + b^2 + c^2}}\label{avplp}} \vs
	}}
\newcommand{\limtpl}{\rg[Linja mellom to plan]{
		Gitt to ikke-parallelle plan, det ene med normalvektor $ \vec{n}_1 $ og det andre med normalvektor $ \vec{n}_2 $. Planene skjærer da hverandre langs ei linje med retningsvektor $ \vec{n}_1\times\vec{n}_2  $.
}}
\newcommand{\vtolin}{\rg[Vinkelen mellom to linjer]{
		Vinkelen $ v $ mellom ei linje med retningsvektoren $ \vec{r}_1 $ og ei linje med retningsvektoren $ \vec{r}_2 $ er gitt ved ligningen
		\begin{equation}
		\cos v = \frac{|\vec{r}_1\cdot\vec{r}_2|}{|\vec{r}_1||\vec{r}_2|} \label{vinklinlin}
		\end{equation}\vs
}}
\newcommand{\vtopl}{\rg[Vinkelen mellom to plan]{Vinkelen $ v $ mellom et plan med normalvektoren $\vec{n}_1$ og et plan med normalvektoren $ \vec{n}_2 $ er gitt ved ligningen
		\begin{equation}\label{vinkmellomtoplan}
		\cos v = \frac{|\vec{n}_1\cdot\vec{n}_2|}{|\vec{n}_1||\vec{n}_2|}
		\end{equation}\vs
}}
\newcommand{\vplli}{\rg[Vinkel mellom plan og linje]{Vinkelen $ v $ mellom et plan med normalvektor $ \vec{n} $ og ei linje med retningsvektor $ \vec{r} $ er gitt som
		\begin{equation}
		v= 90^\circ-w 
		\end{equation}
		hvor $ w $ er gitt ved ligningen
		\begin{equation}
		\cos w = \frac{|\vec{n}\cdot\vec{r}\,|}{|\vec{n}||\vec{r}\,|}
		\end{equation}\vs
}}