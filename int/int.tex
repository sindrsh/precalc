\input{../doc}
\input{../preamb}

\begin{document}
	
\subimport{}{rg}
\eqlen	

\vspace{\parskip}

\textbf{Mål for opplæringen er at eleven skal kunne}
\begin{itemize}
	\item gjøre rede for definisjonen av bestemt integral som grense for en sum og ubestemt integral som antiderivert
	\item beregne integraler av de sentrale funksjonene ved antiderivasjon og ved hjelp av variabelskifte, ved delbrøkoppspalting med lineære nevnere og ved delvis integrasjon
	\item tolke det bestemte integralet i modeller av praktiske situasjoner og bruke det til å beregne arealer av plane områder og volumer av omdreiningslegemer
\end{itemize}
\newpage

\section{Bestemt og ubestemt integral}
\tssec{Bestemt integral \label{bestmint}}
Tenk at vi kjører en bil med en fart som til enhver tid $ t $ er gitt av en funksjon $ v(t) $. Etter en tid $ {t=b }$ ønsker vi å vite lengden $ S $ vi har kjørt siden tiden $ t=a $.\vsk

La oss først si at farten $ v $ er en konstant. Lengden vi har kjørt i tidsintervallet $ [a, b] $ må da være\footnote{$ \text{strekning}=\text{fart}\cdot\text{tid} $}
\[ S = v\cdot(b-a) \]
Figurativt blir dette arealet til firkanten som er avgrenset av $ t $-aksen, linjene $ t=a $, $ t=b $ og grafen til $ v$:
\fig{int0}{\label{fart}}
Men hvordan kan vi finne $ S $ hvis farten er varierende med tiden, som vist i \fref{fartvar}\,?
\fig{int1a}{\label{fartvar}}
Én tilnærming er å plukke ut små intervaller hvor vi regner farten som konstant. Vi starter her med å dele grafen inn i tre like brede intervaller, som da får bredden $ \Delta t=\frac{b-a}{3} $. Videre bruker vi $ v(t) $ i starten av hvert intervall som konstantfart, de tilhørende tidspunktene kaller vi $ {t_1=a} $, $ t_2 $ og $ t_3 $. Vi kan nå anslå $ S $ som summen av tre strekninger $ s_1 $, $ s_2 $ og $ s_3 $ reist med konstant fart:
\alg{S &\approx s_1 + s_2 + s_3 \\
	&\approx v(a)\Delta t + v(t_2)\Delta t + v(t_3)\Delta t \\
	&\approx (v(a) +v(t_2) +v(t_3))\Delta t
	}
Grafisk har vi tilnærmet $ S $ ved å legge sammen arealet av de tre grønne søylene i \fref{binttreint}:
\fig{int1}{\label{binttreint}}
Intuitivt vil vi tenke at jo mindre intervaller vi bruker, jo riktigere blir det å si at farten er konstant over intervallet, og at tilnærmingen da må bli bedre.
\begin{figure}[H]
	\centering
	\subfloat[a)]{\includegraphics[scale=0.9]{\asym{int2}}}\quad
	\subfloat[b)]{\includegraphics[scale=0.9]{\asym{int3}}}	
	\captionof{figure}{\textsl{a)} 10 intervaller \textsl{b)} 20 intervaller \label{toints}}
\end{figure}
Så hvorfor ikke lage uendelig mange, uendelig små\footnote{Med ''uendelig små'' menes det at verdien går mot 0. Størrelser som går mot 0 kalles for \textit{infinitesimale størrelser}\index{infinitesimal}.} intervaller? Vi lar antall intervaller være gitt ved tallet $ n $ og lar $ {n\to \infty} $.
Vi får da at
\begin{align}
	S &\approx\lim\limits_{n\to \infty}(v(t_1)+v(t_2)+...+v(t_{n}))\Delta t \nonumber \\
	&\approx \lim\limits_{n\to \infty}\sum\limits_{i=1}^{n} v(t_{i})\Delta t \nonumber
\end{align}
hvor $ t_i=a+(i-1)\Delta t$ og $ \Delta t=\frac{b-a}{n} $ (legg merke til at $ t_1=a $).

Faktisk kan det faktisk vises\footnote{Se side \pageref{bintforklaring} for en grundigere forklaring.} at: 
\[ S = \lim\limits_{n\to \infty}\sum\limits_{i=1}^{n} v(t_{i})\Delta t \]
I R2 kan vi se på dette som selveste definisjonen av  \textit{det bestemte integralet}\footnote{Hvilke bokstaver vi bruker for å indikere størrelser, funksjoner og variabler er selvsagt helt vilkårlig. I oppsummeringen har vi valgt å bruke de mer klassiske bokstavene $ I $, $ f $ og $ x $ istedenfor $ S $, $ v $ og $ t $.}\index{integral!bestemt} av $ v $ over intervallet $ [a, b] $.\regv
\bint
\eks{
Finn det bestemte integralet av ${ f(x)=x }$ på intervallet $ {x\in[0, 4]} $.

\sv
Vi har her at $ {f(x_i)=x_i=(i-1)\Delta x} $, hvor $ {\Delta x=\frac{4}{n}} $. Setter vi dette inn i \eqref{bint}, får vi at
\alg{
I&= \lim\limits_{n\to \infty}\sum\limits_{i=1}^{n}(i-1)\left(\frac{4}{n}\right)^2 \\
&=4^2 \lim\limits_{n\to \infty} \frac{1}{n^2}\left(\frac{n(n+1)}{2}-n \right) \\
&= 4^2 \lim\limits_{n\to \infty}\frac{1}{n^2} \left(\frac{n^2+n}{2}-n \right) \\
&= 16\frac{1}{2} \\
&= 8
}
\textsl{Merk:} I overgangen mellom første og andre linje i ligningen over har vi brukt summen av en aritmetisk rekke.
}\regv
I kommende seksjoner skal vi finne integraler på en helt annen måte enn i eksempelet over. Læresetningen som sørger for dette er så viktig at den rett og slett kalles \textit{Analysens fundamentalteorem}\footnote{Analyse i matematisk sammenheng kan, kort oppsummert, sies å være studien av funksjoner. Teorem er en læresetning som kan bevises.}. Fordi teoremet gir oss en metode som omgår utregning av summer, lønner det seg å skrive integralet på en mer kompakt form\footnote{Man kan sammenligne dette med å erstatte grensesummen i \eqref{bint} med $ \int\limits_{a}^b $, grenseintervallet med $ dx $, og deretter fjerne alle indekser.}:\regv
\bintto
\label{bestmintend}
\tssec{Analysens fundamentalteorem \label{anfundteo}}	
Tenk igjen at vi kjører med en hastighet gitt av funksjonen $ v(t) $, og at strekningen vi har kjørt nå er gitt ved funksjonen $ s(t) $. I R1 lærte vi at farten\index{fart} er den deriverte av strekningen, altså at:
\[ s'(t)= v(t) \]
Når $ s $ er kjent kan vi enkelt finne den totale strekningen $ S $ vi har reist på intervallet $ t\in[a, b] $:
\[ S = s(b)-s(a) \]
Men som vi har sett kan $S $ også beskrives som et bestemt integral:
\[ S = \int\limits_{a}^b v(t)\,dt \]
Denne sammenhengen kan generaliseres til å gjelde for alle kontinuerlige funksjoner:\regv
\anfndto
\anfndtoe
\tssec{Ubestemte integral}\index{integral!ubestemt}
Vi har hittil sett på det \textit{bestemte} integralet, som har sitt navn fordi integralet er over et intervall der start- og sluttverdien er gitt. Det \textit{ubestemte} integralet til en funksjon $ f(x) $ skriver vi derimot som
\[ \int\limits_c^x f(t)\, dt \]
Navnet ubestemt kommer av at $ c $ er en vilkårlig konstant og at $ x $ er en varierende verdi\footnote{Det kan kanskje se litt rart ut at vi har skrevet $ f(t) $ i integralet når vi snakker om $ f(x) $, men dette gjøres bare for å skille mellom de to varierende verdiene $ x $ og $ t $. $ x $ kan være en hvilken som helst verdi, men for det ubestemte integralet ser vi på $ f $ for verdiene $ {t\in [a, x]} $, altså $ f(t) $. Og da er det ikke $ x $ som varierer, men $ t $, derav $ dt $.}.\vsk

Hvis vi lar $ F$ være en antiderivert til $ f $, har vi fra (\ref{anfund2}) at:
\[ \int\limits_c^x f(t)\, dt = F(x)-F(c)  \]
Siden $ c $ er en konstant, må $ -F(c) $ også være det. Denne kalles \textit{integrasjonskonstanten} og omdøpes gjerne til $ C $. Det er også vanlig å forenkle skrivemåten til det ubestemte integralet ved å fjerne grensene og bare skrive $ f(x)\,dx $ etter integraltegnet.\regv
\uint\regv
\textsl{Merk}: Når ikke annet er nevnt, tar vi det heretter for gitt at størrelser skrevet som store bokstaver er vilkårlige konstanter som resultat av integrasjon.\regv
\uinte
\newpage
\uinteto


\begin{comment}
	\tssec{Analysens fundamentalteorem I}	
	Når vi har etablert konseptet bak det ubestemte integralet, er veien kort til den andre delen av analysens fundamentalteorem. Vi deriverer da på begge sider av (\ref{uint}):
	\[ \left(\int f(x)\, dx\right)'  = \left(F(x)+ C\right)' \]
	Vi vet at $ F'(x)=f(x) $, og videre er $ C'=0 $, derfor får vi:
	\anfnden
	\anfndene
\end{comment}
\section{Integralregning}
Å finne bestemte og ubestemte integraler er et stort og viktig felt innenfor matematikken. Analysens fundamentalteorem forteller oss at nøkkelen er å finne en antiderivert til funksjonen vi ønsker å integrere.
\tssec{Integralet av utvalge funksjoner \label{intuf}}
Vi skal etterhvert se at å finne integraler ofte krever spesielle metoder, men noen grunnleggende relasjoner bør vi huske:
\newpage
\uints
\uintse 
\eks[2]{
Finn det ubestemte integralet $ \displaystyle \int \left(\frac{1}{x^4}+\sqrt[3]{x}\right)dx $.

\sv
Vi utnytter at $ {\frac{1}{x^4}=x^{-4}} $ og at $ {\sqrt[3]{x}=x^{\frac{1}{3}}} $. Ved (\ref{intfplusg}) og (\ref{xhatr}) kan vi skrive:
\alg{
\int \left(\frac{1}{x^4}+\sqrt[3]{x}\right)dx &= \int \left(x^{-4} +x^{\frac{1}{3}}\right)\,dx \\
&= \frac{1}{-4+1}x^{-4+1}+\frac{1}{\frac{1}{3}+1}x^{\frac{1}{3}+1}+C \br
&= -\frac{1}{3}x^{-3}+\frac{3}{4}x^{\frac{4}{3}}+C
}
\vsb
} \newpage
\tssec{Bytte av variabel}
Vi skal nå se på en metode som kalles \textit{bytte av variabel}\footnote{Det er flere framgangsmåter for denne metoden. Den vi her presenterer er, etter forfatterens mening, den raskeste for integraler som er pensum i R2. For mer avanserte integraler bør man kjenne til framgangsmåten presentert i \hrv{intleibn}.}\index{integrasjon!bytte av variabel} (også kalt \textit{substutisjon}). Med denne kan vi ofte forenkle integralregningen betraktelig.\regv
\byt
\byte
\newpage
\byteto
\eks[3]{
	Buelengden til grafen til en funksjon $ f(x) $ på intervallet $ [a, b] $ er gitt som
	\[ \int\limits_a^b\sqrt{1+(f')^2}\,dx \tag{I}\label{Ieks3} \]
	Finn lengden til funksjonen
	\[ f(x)=\frac{1}{3}x^\frac{3}{2}\qquad,\qquad x\in[0, 5] \]
	
	\sv
	Vi har at
	\algv{
	f'= \frac{1}{2}x^{\frac{1}{2}}	
	}
Og videre at
\alg{
	(f')^2=\frac{1}{4}x
}
	Det ubestemte integralet i \eqref{Ieks3} blir da
	\alg{
	\int\limits\sqrt{1+\frac{1}{4}x}\,dx		}
Vi setter $ u=1+\frac{1}{4}x $ og $ g(u)=u^\frac{1}{2} $. Da er $ u'=\frac{1}{4} $. Nå har vi at
	\alg{
	\int\limits\sqrt{1+\frac{1}{4}x}\,dx
	&= 4\int  u^{\frac{1}{2}} u'\,dx \\
	&=4\int u^{\frac{1}{2}}\,du \br
	&=\frac{8}{3}u^{\frac{3}{2}}+C\br
	&=\frac{8}{3}\left(1+\frac{1}{4}x\right)^{\frac{3}{2}}+C
}
Altså er
\alg{
\int\limits_0^5 \sqrt{1+(f')^2}\,dx&=	\frac{8}{3}\left[\left(1+\frac{1}{4}x\right)^{\frac{3}{2}}\right]_0^{5}\br
&=\frac{8}{3}\left(\left(1+\frac{5}{4}\right)^{\frac{3}{2}}-1\right) \br
&=\frac{8}{3}\left(\left(\frac{9}{4}\right)^{\frac{3}{2}}-1\right) \br
&=\frac{8}{3}\left(\frac{27}{8}-1\right)\br
&=\frac{19}{3}
}
\textsl{Merk:} En litt lettere utrekning kunne vi fått ved å observere at
\[ \sqrt{1+\frac{1}{4}x}=
 \frac{1}{2}\sqrt{4+x} \]
Med denne omskrivingen kunne vi valgt substutisjonen \\$ u=4+x $, og dermed fått at $ u'=1 $.
}

\tssec{Delvis integrasjon}
Hvis vi ikke finner et passende bytte av variabel for å løse et integral, kan vi isteden prøve med \textit{delvis integrasjon}\index{integrasjon!delvis}. Vi starter med å utlede ligningen som legger grunnlaget for metoden.\vsk

Gitt produktet av to funksjoner $ u(x) $ og $ v(x) $, altså $ uv $. Av produktregelen ved derivasjon (se (\ref{prreg})) har vi at
\[ (uv)'=u'v+uv' \]
Videre integrerer\footnote{Når vi har flere ubestemte itegraler, trenger vi bare ta med integrasjonskonstanten for én av dem. Derfor er ikke konstanten fra integrasjonen av $ (uv)' $ tatt med.} vi begge sider av ligningen over mhp. $ x $:
\alg{
	\int (uv)' \, dx &=\int \left(u'v+uv'\right) \, dx \\
	uv &= \int \left(u'v+uv'\right) \, dx \\
	uv -\int u'v \, dx &= \int uv' \, dx
}
\dint
\dinte
\newpage
\dinteto
\tssec{Delbrøksoppspaltning}
Gitt integralet
\[\int  \frac{4x+5}{(x+1)(x+2)} \,dx \]
Etter litt testing vil vi finne at både delvis integrasjon og bytte av variabel kommer til kort i vår søken etter en antiderivert. Hva vi heller kan gjøre, er å ta i bruk \textit{delbrøksoppspaltning}\index{integrasjon!delbrøksoppspaltning}. \vsk

Vi merker oss da at integranden\index{integrand}\footnote{For $ \int f(x)\,dx $ sier vi at $ f $ er \textit{integranden}.} er en brøk med nevneren $ (x+1)(x+2) $. Dette betyr at den kan skrives som to separate brøker med $ (x+1) $ og $ (x+2) $ som nevnere:
\begin{equation}
\frac{4x+5}{(x+1)(x+2)} = \frac{A}{x+1}+\frac{B}{x+2} \label{delbr}
\end{equation}
$ A $ og $ B $ er to konstanter, vår oppgave blir nå å bestemme verdien til disse.\vsk

Vi starter med å gange begge sider av (\ref{delbr}) med fellesnevneren:
\alg{
	\frac{4x+5}{(x+1)(x+2)}(x+1)(x+2) &= \left(\frac{A}{x+1}+\frac{B}{x+2}\right)(x+1)(x+2) \br
	4x+5 &= A(x+2)+B(x+1)
	}
For det rette valget av $ A $ og $ B $ er uttrykkene over like for alle verdier av $ x $. Når $ {x=-1} $, har vi bare $ A $ som ukjent:
\alg{
	4\cdot(-1)+5 &= A(-1+2)+B(-1+1)\\
	1 &= A	
	}
Og ved å sette $ x=-2 $, finner vi $ B $:
\alg{
	4\cdot(-2) +5&= A(-2+2)+B(-2+1) \\
	-3 &= -B \\
	3 &= B	
	}
Nå kan vi altså skrive
\[ \frac{4x+5}{(x+1)(x+2)} = \frac{1}{x+1}+\frac{3}{x+2}  \]
Dette er to brøker vi kan å integrere\footnote{\textsl{Obs!} I søken etter $ A $ og $ B $ valgte vi verdiene ${ x=-1 }$ og $ {x=-2} $. I ligningene hvor vi satte inn disse verdiene var dette helt uskyldig, men i integralet må vi være observante. Vi får nemlig 0 i nevner hvis én av disse verdiene ligger i intervallet vi skal integere over. Er det snakk om et bestemt integral må vi derfor passe på at dette ikke er tilfelle.} (se (\ref{lnint})):
\alg{
\int \frac{4x+5}{(x+1)(x+2)}\,dx &= \int \left(\frac{1}{x+1}+\frac{3}{x+2}\right)\,dx \\
&= \ln |x+1| + 3\ln|x+2| 	
	}
\newpage
\dbr
\dbre
\eks[2]{\label{delbre2}
Finn det ubestemte integralet
\[ \int \frac{x^3 + 5 x^2 + x - 4}{x^2+x-2}\,dx\] \vs
\sv

Hvis telleren har potenser av høyere orden\footnote{Her har telleren tre som høyeste orden, mens nevneren har to.} enn nevneren, må vi starte med en polynomdivisjon:

\begin{align*}
\phantom{-}&(x^3 + 5 x^2 + x - 4):(x^2+x-2) =x+4+\frac{-x+4}{x^2+x-2} \\ 
-&\underline{(x^3+x^2-2x)} \\
&\phantom{(x^3)}4x^2+3x-4 \\
&\phantom{\,}-\underline{(4x^2+4x-8)}\\
&\phantom{aaaaaaa\,}-x+4
\end{align*}
Vi observerer videre at nevneren i brøken kan omskrives til $ (x-1)(x+2) $, for to konstanter $ A $ og $ B $ har vi altså at
\alg{
	\frac{A}{x-1}+\frac{B}{x+2}&=\frac{-x+4}{x^2+x-2}  \br
	A(x+2)+B(x-1)&=-x+4
}
Når $ {x=-2} $, får vi at
\alg{
	B(-2-1)&=-(-2)+4 \\
	B &= -2
}
Og når ${ x=1} $, er
\alg{
	A(1+2) &= -1+4 \\
	A &= 1
}
Integralet blir derfor
\[ \int \left(x+4+\frac{1}{x-1}-\frac{2}{x+2}\right)\,dx
= \frac{1}{2}x^2+4x + \ln |x-1|-2\ln|x+2|+C \] \vs
}
\begin{comment}
\begin{multline*}
\int \left(x+4+\frac{1}{x-1}-\frac{2}{x+2}\right)\,dx
= \frac{1}{2}x^2+4x + \ln (x-1)\\-2\ln(x+2)+C
\end{multline*}
\end{comment}
\section{Areal og volum}
\tssec{Avgrenset areal}
Som antydet i \hrs[delseksjon]{bestmint} er det en sterk sammenheng\footnote{Se s. \pageref{bintforklaring}-\pageref{bintslutt} for nærmere forklaring.} mellom det bestemte integralet av en funksjon $ f(x) $ på intervallet $ [a, b] $ og arealet avgrenset av grafen til $ f $, $ x $-aksen og linjene $ {x=a} $ og $ {x=b} $. Sistnevnte størrelse skal vi for enkelhetsskyld kalle \textsl{arealet avgrenset av $ f $ for $x\in [a, b] $}:\newpage
\iar \vsk
\textbf{Areal avgrenset av to funksjoner}\bs
Noen ganger ønsker vi også å finne arealet avgrenset av to funksjoner. Da må vi sørge for at vi har tilstrekkelig med informasjon om disse før vi utfører integrasjonen:\regv
\iarto \newpage
\eks{
Gitt funksjonene $ {f(x)=\sin\left(\frac{\pi}{2}x\right)} $ og $ {g(x)=2x-1} $. Vi har da at $ {f\geq g} $ for ${ x\leq 1} $ og at $ {g\geq f} $ for $ {x\geq1} $. Finn arealet $ A $ avgrenset av $ f $ og $ g $ for $ x\in [0, 2] $.  \\

\sv
Ut ifra informasjonen over er arealet gitt ved ligningen
\alg{
A &= \int\limits_{0}^{1} (f-g)\,dx + \int\limits_{1}^{2} (g-f)\,dx 
}
Vi starter med å regne ut de to integralene hver for seg:
\alg{
\int\limits_{0}^{1} (f-g)\,dx&= \left[-\frac{2}{\pi}\cos\left(\frac{\pi}{2}x\right)-(x^2-x)\right]_0^1\\
&= -\left[\frac{2}{\pi}\cos\left(\frac{\pi}{2}x\right)+(x^2-x)\right]_0^1\br
&= \frac{2}{\pi}\cos\left(\frac{\pi}{2}\cdot1\right)+(1^2-1) \\ &\qquad\qquad-\left(\frac{2}{\pi}\cos\left(\frac{\pi}{2}\cdot0\right)+(0^2-0)\right)\\
&= \frac{2}{\pi}
} \vds
\alg{
\int\limits_{1}^{2} (g-f)\,dx&=\left[(x^2-x)+\frac{2}{\pi}\cos\left(\frac{\pi}{2}x\right)\right]_1^2\\
&=(2^2-2)+\frac{2}{\pi}\cos\left(\frac{\pi}{2}\cdot2\right) \\ &\qquad\qquad-\left(-\frac{2}{\pi}\cos\left(\frac{\pi}{2}\cdot1\right)-(1^2-1)\right) \\
&=2 -\frac{2}{\pi}
}
Summen av disse to integralene er $ 2 $, som altså er arealet.
}
\newpage
\tssec{Volumet av en figur}
Vi har sett hvordan integraler kan brukes til å finne arealer, men de kan også brukes til å finne volumer:\regv
\ivo
\ivoe
\newpage
\tssec{Volum av omdreiningslegemer}
Si vi har en funksjon $ f(x) $ gitt på intervallet $ [a, b] $, med en graf som vist i \fref[a]{omdr}. Tenk nå at vi dreier linjestykket $ 360^\circ $ om $ x $-aksen. Formen vi da har ''skjært'' ut, vist i \fref[b]{omdr}, er det vi kaller \textit{omdreiningslegemet}\index{omdreiningsleme} til $ f(x) $ på intervallet $ [a, b] $.
\begin{figure}[H]
	\centering
\subfloat[a)]{
	\includegraphics[]{\asym{omdr2}}
} \\
\subfloat[b)]{
	\includegraphics[]{\asym{omdr3}}
	}
	\captionof{figure}{\textsl{a)} Grafen til $ f $. \textsl{b)} Omdreiningslegemet til $ f $. \label{omdr}}
\end{figure}
Tverrsnittet (langs $ x $-aksen) til en slik figur er alltid sirkelformet,\\ tverrsnittsarealet er derfor $ \pi r^2 $, hvor $ r(x) $ er radiusen til tverrsnittet. Men siden radiusen tilsvarer høyden fra $ x $-aksen opp til $ f $, er $ {r=f}$. Av (\ref{intvol}) kan vi da skrive
\[ V = \int\limits_a^b A \,dx 
=\int\limits_a^b \pi f^2 \,dx =
\pi\int\limits_a^b f^2 \,dx \]
\omdr
\newpage
\omdre
\newpage
\tsec{Forklaringer}
\subsection*{Bestemt integral \label{bintforklaring}}
På side \pageref{bestmint}\,-\,\pageref{bestmintend} brukte vi en funksjon $ v(t) $ 
som ga oss en hastighet for enhver tid $ t $. 
Vi presenterte da integralet som en tilnærming 
av hvor langt man hadde beveget seg over et tidsintervall $ {t\in[a, b]} $. Når vi nå skal studere integralet helt generelt, starter vi isteden med en geometrisk definisjon av integralet:\vsk

\textsl{Gitt en funksjon $ f(x) $ som er positiv og kontinuerlig for alle $ {x\in[a, b] }$. Integralet $ I $ tilsvarer arealet avgrenset av $ x $-aksen, linjene $ {x=a} $ og ${ x= b} $ og grafen til $ f $.}
\fig{int5}{\label{Ifig} Integralet $ I $ tilsvarer det avgrensede arealet i grønt.}
La oss ta utgangspunkt i funksjonen $ f(x) $, med en graf som vist i \fref{Ifig}. Vårt mål er nå å finne $ I $.\vsk

Vi starter med å splitte $ [a, b] $ inn i $ n $ mindre delintervaller, alle med bredden $ {\Delta x = \frac{b-a}{n}} $. I tillegg lar vi $ x_i $ for $ {i\in\lbrace 1, 2,\,...\,, n\rbrace} $ betegne den $ x $-verdien som er slik at $ f(x_i) $ er den laveste verdien til $ f$ på delintervall nr. $ i $. \vsk

Arealet avgrenset av delintervallet og $ f $ tilnærmer vi som $s_i= f(x_i)\Delta x $, da har vi at (se \fref{bintforkl})
\alg{
	I &\geq s_1+s_2+...+s_i\\
	I&\geq f(x_1)\Delta x + f(x_2)\Delta x +...+f(x_n)\Delta x  \\
	I &\geq \sum\limits_{i=i}^{n} f(x_i)\Delta x	
}
\fig{int4}{Arealene av $ s_i $ markert som grønne søyler og arealene av $ c_i $ markert som blå søyler. Bredden til hver søyle er $ \Delta x = \frac{b-a}{n} $ (her er \y{n=4 }).\label{bintforkl}}
Videre må det finnes et tall $ {h_i\in [0, 1)} $ som er slik at $ f(x_i+h_i\Delta x) $ er den høyeste verdien til $ f $ på delintervallet. Vi lar $ c_i $ betegne arealet til søylen med $ \Delta x $ som bredde og  $ f(x_i+h_i\Delta x)-f(x_i)$ som høyde:
\[ c_i = (f(x_i+h_i\Delta x)-f(x_i))\Delta x \]
Hvis vi legger til alle $ c_i $ i det første estimatet vårt, får vi en tilnærming som må være større eller lik det egentlige arealet. Derfor kan vi skrive
\[\sum\limits_{i=1}^{n} f(x_i)\Delta x \leq I \leq \sum\limits_{i=1}^{n} f(x_i)\Delta x+ \sum\limits_{i=1}^{n} c_i \]
Én av $ c $-verdiene må være større eller lik alle andre $ c $-verdier. Vi lar $ m $ betegne indeksen til nettopp denne $ c $-verdien. Da må vi ha at
\[0\leq \sum\limits_{i=1}^{n} c_i \leq nc_{m} \]  
Men når \y{n\to\infty}, går summen $ nc_m $ mot 0:
\alg{
	\lim\limits_{n\to\infty}  n c_{m}&=\lim\limits_{n\to\infty} n(f(x_{m}+h_{m}\Delta x)-f(x_{m}))\Delta x	\\
	&= \lim\limits_{n\to\infty} n(f(x_{m}+h_{m}\Delta x)-f(x_{m}))\frac{b-a}{n} \\
	&= \lim\limits_{n\to\infty} (f(x_{m}+h_{m}\Delta x)-f(x_{m}))(b-a)\\
	&= \lim\limits_{n\to\infty} (f(x_{m})-f(x_{m}))(b-a) \\
	&= 0	
	}
\newpage
Følgelig er $ \lim\limits_{x\to\infty} \sum\limits_{i=1}^{n} c_i =0,$ og da er
\begin{align}
	\lim\limits_{n\to\infty}\sum\limits_{i=1}^{n} f(x_i)\Delta x \leq &I \leq \lim\limits_{n\to\infty}\left(\,\sum\limits_{i=1}^{n} f(x_i)\Delta x+ \sum\limits_{i=1}^{n} c_i\right) \label{bintforkl1} \br
	\lim\limits_{n\to\infty}\sum\limits_{i=1}^{n} f(x_i)\Delta x \leq &I\leq \lim\limits_{n\to\infty}\sum\limits_{i=1}^{n} f(x_i)\Delta x \\
	&I= \lim\limits_{n\to\infty}\sum\limits_{i=1}^{n} f(x_i)\Delta x	\label{bintforkl2}
\end{align}
Det vi har kommet fram til nå er vel og bra, men skal vi regne ut et integral blir det slitsomt å inspisere $ f(x) $ på uendelig mange delintervaller for å finne de laveste funksjonsverdiene i hver av dem! Vi merker oss derfor at venstresiden i (\ref{bintforkl1}), i vårt tilfelle, representerer det kraftigste underestimatet av $ I $, mens høyresiden er det kraftigste overestimatet. I (\ref{bintforkl1})-(\ref{bintforkl2}) har vi vist at begge disse estimatene går mot $ I $ når $ {n\to\infty} $, dette betyr at vi for andre valg av $ x_i $ på hvert intervall også kommer fram til ønsket resultat. Regneteknisk vil det ofte være lurt å velge $ x_i=a+(i-1)\Delta x $ for $ {i\in\lbrace 1, 2,\,...\,, n\rbrace} $, slik som i (\ref{bint}).\vsk

\textbf{Integral som areal for negative funksjoner}\bs
Hva nå om vi isteden skulle finne arealet avgrenset av $ x $-aksen, linjene $ {x=a} $ og $ {x= b }$ og grafen til $ {g(x) = -f(x)} $? \vsk

Grafene til $ f $ og $ g $ er fullstendig symmetriske om $ x$-aksen, dette må bety at arealet $ A $ de avgrenser på et intervall må være helt likt. Og vi vet at\vs
\alg{
	A &= \lim\limits_{n\to\infty}\sum\limits_{i=1}^{n} f(x_i)\Delta x \\
	&= 	\lim\limits_{n\to\infty}\sum\limits_{i=1}^{n} -g(x_i)\Delta x \\
	&= -\lim\limits_{n\to\infty}\sum\limits_{i=1}^{n} g(x_i)\Delta x
}
Av dette kan vi utvide den geometriske definisjonen av integralet:\vsk

 \textsl{Gitt en funksjon $ f(x) $ som er negativ og kontiunerlig for alle $ {x\in[a, b] }$. Integralet $ I $ multiplisert med $ -1 $ tilsvarer arealet avgrenset av $ x $-aksen, linjene $ {x=a} $ og ${ x= b} $ og grafen til $ f $.}
\label{bintslutt}

\subsection*{Analysens fundamentalteorem}
Vi ønsker å vise at integralet $ I $ av en funksjon $ f(x) $ på intervallet $ [a, b] $ er gitt som
\[ I = F(b)-F(a) \]
hvor $ F $ er en antiderivert til $ f $. For å vise dette skal vi anvende oss av \eqref{bint}. Spesielt verdt å merke seg er at $ x_1=a $ og at $ x_{n+1}=b $.\vsk

Fra tidligere vet vi at den deriverte av en funksjon $ f(x) $ er gitt som
\[ f'(x) = \lim\limits_{\Delta x\to 0}\frac{f(x+\Delta x)-f(x)}{\Delta x} \]
Med vår $ \Delta x=\frac{b-a}{n} $ kan vi omskrive grensen:
\[  f'(x)=\lim\limits_{n\to \infty}\frac{f(x+\Delta x)-f(x)}{\Delta x} \]
La $ F(x) $ være en antiderivert til $ f(x) $, da er
\[ F'(x)=f(x)=\lim\limits_{n\to \infty}\frac{F(x+\Delta x)-F(x)}{\Delta x}  \]
Vi erstatter $ f $ i \eqref{bint} med uttrykket over, og får at
\alg{I=
	\lim\limits_{n\to \infty}\sum\limits_{i=1}^{n} \frac{F(x_i+\Delta x)-F(x_i)}{\Delta x} \Delta x 
}
Fordi $ {x_{i+1}=x_i+\Delta x} $, har vi videre at
\alg{
	I&=	\lim\limits_{n\to \infty}\sum\limits_{i=1}^{n} \left(F(x_{i+1})-F(x_i)\right) \\
	&= \lim\limits_{n\to\infty} \left(F(x_2)-F(x_1)+F(x_3)-F(x_2)+...+F(x_{n+1})-F(x_{n})\right)
}
Av dette legger vi merke til at alle $ F(x_i) $ kansellerer hverandre, bortsett fra i endepunktene. Vi sitter altså igjen med summen
\alg{
	I &=\lim\limits_{n\to\infty}\left(-F(x_1)+ F(x_{n+1})\right) \\
	&= F(b)-F(a)
}

\begin{comment}
	Sammenhengen vi akkurat har studert for integral og areal er sentral når vi nå skal gå over til å forstå konseptet bak \textit{Analysens fundamentalteorem}, som har enorm betydning for integralregning.
	
	Tenk nå at vi har en funksjon $ A(x)=\int\limits_{a}^x f(t)\,dt $ hvor $ f(t)\geq 0$. $ A $ gir oss da arealet avgrenset av horisontalaksen, grafen $ f $ og linjene $ t=a $ og $ t=x $. Tenk så at vi flytter oss en liten avstand $ \Delta x $ fra $ x $, det avgrensede arealet blir da $ A(x+\Delta x) $. Forskjellen $ \Delta A $ mellom $ A(x) $ og $ A(x+\Delta x) $ blir:
	\[ \Delta A = A(x+\Delta x)-A(x) \]
	\subimport{fig/}{int4}
	Fra \fref{fund} kan vi nå skrive:
	\[ \Delta A = f(x)\Delta x + c	 \]
	hvor $0\leq c\leq |(f(x+\Delta x)-f(x))\Delta x|$.
	
	Dersom vi nå flytter over $ c $, deler på begge sider med $ \Delta x$ og finner grensen når $ \Delta x \to 0$, får vi:
	\alg{
	\lim\limits_{\Delta x\to 0}\left[\frac{\Delta A}{\Delta x} - \frac{g(x)}{\Delta x}\right] &= f(x)
	}
	Det kan vises (prøv gjerne selv) at $ \lim\limits_{\Delta x\to 0}\frac{c}{\Delta x} =0$, og vi har da at:
	\alg{\lim\limits_{\Delta x\to 0} \frac{\Delta A}{\Delta x} &= f(x) \\[5 pt]
	A'(x)&=f(x) 	
	}
\end{comment}
\newpage
\subsection*{Integralet av utvalgte funksjoner}
(\ref{intfplusg}) og (\ref{konstint}) følger direkte av (\ref{sumreg1}) og (\ref{sumreg2}).\vsk

Ut ifra definisjonen av det ubestemte integralet (se (\ref{uint})) har vi at
\[ \int f(x)\,dx = F(x)+C \]
hvis $ {F'=f}  $. For alle ubestemte integraler gitt i (\ref{xhatr})-(\ref{tanint}) kan dette sjekkes via enkle derivasjonoperasjoner og er derfor overlatt til leseren.


\subsection*{Bytte av variabel}
Gitt en funksjon $ F(x) $ som vil anta samme verdier som $ G(u(x)) $:
\begin{equation}
 F(x)=G(u)  
\end{equation}
La oss nå skrive $ F'(x) $ som $ f(x) $ og $ G'(u) $ som $ g(u) $. For to konstanter $ C $ og $ D $ må vi ha at
\alg{
\int f(x)\, dx &= F(x)+C \\
 &\text{og} \\
\int g(u)\, du &= G(u)+D 
} Det må derfor finnes en konstant $ E $ som er slik at
\[ \int f(x)\, dx+E=\int g(u)\, du\]

Men av kjerneregelen (\ref{kjerne}) har vi følgende relasjon:
\[ f(x)= g(u)u'   \]
Vi kan derfor skrive
\[ \int g(u)u' \, dx+E =\int g(u)\, du\]
Når vi utfører integrasjonen på enten venstre eller høyre side, får vi en ny konstant som vi kan slå sammen med $ E $. I praksis kan vi derfor utelate $ E $, noe som er gjort i (\ref{bytvar}).


\subsection*{Volumet av geometrier}
Vi setter geometrien vår inn i et koordinatsystem, og tar for gitt at vi har en funksjon $ A(x) $ som gir oss tverrsnittsarealet for alle gyldige $ x $.
\fig{test22}{Volumet av geometrien (gul) tilnærmes ved summen av hver $ A(x_i)\Delta x $ (blå).\label{volgjen}}
Vi deler $ [a, b] $ inn i $ n $ delintervaller, der hvert intervall har lengden $ {\Delta x=\frac{b-a}{n}} $ og startverdi $ {x_i=a+(i-1)\Delta x}$ for $ {i\in\lbrace 1, 2, ...\, , n\rbrace }$. Vi tilnærmer volumet til geometrien ved å legge sammen volumene på formen $ A(x_i)\Delta x  $. Når vi lar $ n $ gå mot uendelig vil summen gå mot volumet til gjenstanden\footnote{Argumentasjonen for denne påstanden blir identisk med den gitt i forklaringen for det bestemte integralet (se side \pageref{bintforklaring}).}, dette kan vi skrive som
\alg{
V &=\lim\limits_{x\to \infty} \left(A(x_0)\Delta x + A(x_1)\Delta x +  ... +A(x_{n})\Delta x\right) \\
&=\lim\limits_{x\to \infty} \left(A(x_0)+ A(x_1) +  ... +A(x_{n})\right)\Delta x \\
&= \lim\limits_{x\to \infty} \sum\limits_{i=1}^{n} A(x_i) \Delta x
	}
Uttrykket over er analogt til definisjonen av det bestemte integralet fra ligning (\ref{bint}).
\end{document}