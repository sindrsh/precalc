\newcommand{\ari}{\rg[Aritmetisk følge]{
		Et ledd $ a_i $  i en aritmetisk følge er gitt ved den rekursive formelen
		\nreq{a_i=a_{i-1}+d }
		og den eksplisitte formelen
		\begin{equation}
		a_i=a_1+d(i-1) \label{ekspl}
		\end{equation}
		hvor $ d $ er den konstante differansen $ a_i-a_{i-1} $.}}
\newcommand{\arie}{\eks[]{Finn den rekursive og den eksplisitte formelen til følgen
		\[ 7, 13, 19, 25, ... \]	\vs
		\sv
		Følgen har konstant differanse $ {d=6} $ og første ledd $ {a_1=7} $. Den rekursive formelen blir da
		\[ a_i = a_{i-1}+6 \]
		Mens den eksplisitte formelen blir
		\[ a_i=7+6(i-1) \]\vds
}}
\newcommand{\geo}{\rg[Geometrisk følge]{
		Et ledd $ a_i $  i en geometrisk følge med kvotient $ k $ er gitt ved den rekursive formelen
		\nreq{a_i=a_{i-1}\cdot k}
		
		og den eksplisitte formelen
		\nreq{a_i=a_1\cdot k^{i-1}}\vds
	}}
\newcommand{\sar}{\rg[Summen av en aritmetisk rekke]{
		Summen $ S_n $ av de $ n $ første leddene i en aritmetisk rekke er gitt som
		\nreq{S_n = n\frac{a_1+a_n}{2} \label{arrek}}
hvor $ a_1 $ er første ledd i rekken.	}
}
\newcommand{\sare}{\eks{
		Gitt den uendelige rekken
		\[ 3+7+11+... \] 
		\textbf{a)} Finn summen av de ti første leddene. \os
		
		\textbf{b)} For hvilken $ n $ er summen av rekken lik 903?\\
		
		\sv
		\textbf{a)} Det $ i $-te leddet $a_i$ i rekken er gitt ved formelen
		\[ a_i=3+4(i-1) \]
		Dette er derfor en aritmetisk rekke, og summen av de $ n $ første  leddene er da gitt av ligning (\ref{arrek}). Ledd nr.\,10 blir da\footnote{Når én side av ligningen er blank, betyr dette at uttrykket på siden er uforandret.}
		\alg{
			a_{10} &= 3+4(10-1) \\
			&= 39	
		} 
		De ti første leddene er dermed gitt som
		\alg{
			S_{10} &= 10\cdot\frac{3+39}{2} \\
			&= 210
		}
		\textbf{b)} I formelen for $ S_n $ setter vi inn det eksplisitte uttrykket for $ a_n $, og får at	
		\alg{ S_n &= n\frac{a_1+a_1+d(n-1)}{2} \\
			2\cdot903 &= n(3+3+4(n-1))\\
			0 &= 6n+4n^2-4n - 2\cdot903 \\
			0 &= 2n^2+n-903
			}
		Denne ligningen har løsningene $ {n\in \big\lbrace21, -\frac{43}{2} \big\rbrace}$. Vi søker et positivt heltall, derfor er $ {n=21} $ eneste mulige løsning.
	}}
\newcommand{\sge}{\rg[Summen av en geometrisk rekke \label{geom}]{Summen $ S_n $ av de $ n $ første leddene i en geometrisk rekke med kvotient $ k $ og første ledd $ a_1 $ er gitt som
		\nreq{S_n=a_1\frac{1-k^n}{1-k}\quad, \quad k\neq 1 \label{sumg}}
		Hvis $ {k=1}, $ er
		\nreq{ S_n = na_1}\vds
	}}
\newcommand{\suge}{\rg[Summen av en uendelig geometrisk rekke]{
		For en uendelig geometrisk rekke med kvotient $ {k<|1|} $ og første ledd $ a_1 $ er summen $ S_\infty $ av rekken gitt som
		\nreq{S_\infty = \frac{a_1}{1-k}}
		
		Hvis $ |k|\geq 1 $, vil summen gå mot $ \pm \infty $.
	}}
\newcommand{\ind}{
\rg[Induksjon]{
	Når vi ved induksjon ønsker å vise at ligningen
	\begin{equation}\label{induksjon}
	A(n)=B(n)
	\end{equation}
	er sann for alle $ n\in\mathbb{N} $, gjør vi følgende:
	\begin{enumerate}
		\item Sjekker at \eqref{induksjon} er sann for $ {n=1} $.
		\item Sjekker at \eqref{induksjon} er sann for $ {n=k+1} $, antatt at den er sann for $ {n=k} $.
	\end{enumerate}
}
}
