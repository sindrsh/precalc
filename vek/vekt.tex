\input{../doc}
\input{../preamb}

\begin{document}
\subimport{}{rg}
\eqlen	
%\setcounter{chapter}{2}
%\tableofcontents \newpage
%\chapter{Vektorer i rommet}
\begin{comment}
	I matematikk R1 studerte vi vektorer i planet og brukte da et koordinatsystem med en $ x $- og en $ y $-akse. Vi skal nå gå et skritt videre og innføre også en $ z $-akse som står vinkelrett på de to andre. Vi får da en vektor som kan gå i tre retninger, og vi sier da at vi befinner oss i \textit{rommet}. 
	
	Om vi ønsker å studere krefter som virker på gjenstander er det vektorer i rommet vi bruker for å beskrive disse. Dette emnet spiller derfor en stor rolle uansett om det er bevegelsen til bakterier, biler eller fly man ønsker å undersøke. \\
\end{comment}

\textbf{Mål for opplæringen er at eleven skal kunne}
\begin{itemize}
	\item utføre beregninger med tredimensjonale vektorer som er representert både geometrisk og på koordinatform
\item bruke og tolke skalar- og vektorproduktet i beregning av avstander, vinkler, areal og volum
\end{itemize}
\newpage	
\section{Vektorbegrepet}\index{vektor}
For å beskrive størrelser i rommet, innfører vi et koordinatsystem der en $ x$- en $ y $- og en $ z $-akse står vinkelrett på hverandre. En vektor i rommet vil angi en lengde langs hver av disse aksene La oss bruke vektoren $ \vec{u}=[2, 3, 4] $ som eksempel. Når vi skriver $ \vec{u} $ på denne måten, sier vi at vektoren er skrevet på \textit{komponentform}\index{komponent} med 2 som $ x $-komponent, 3 som $ y $-komponent og 4 som $ z $-komponent. \vsk

For å framstille $ \vec{u} $ grafisk tegner vi en pil fra et startpunkt til punktet vi når ved å følge lengdene\footnote{Hvis en komponent er negativ betyr dette at lengden skal vandres motsatt av akseretningen.} som vektoren angir.
\fig{vekt1}{$ \vec{u}=[2, 3, 4] $\label{vekt1}}
I figuren over er $ \vec{u} $ vist i det som kalles \textit{grunnstillingen}\index{grunnstilling}, dette innebærer at den starter i origo. Enhver vektor som starter i grunnstillingen vil ende i punktet med samme koordinater som vektorens komponenter. Hvis man følger en vektor fra et punkt til et annet, har man bevegd seg i en bestemt retning, pilen som tegnes peker i netopp denne retningen\index{retning}. \newpage

\subsection{Vektoren mellom to punkt}\index{vektor!mellom to punkt}
Det er viktig å innse at en vektor kan befinne seg hvor som helst i koordinatsystemet. Trekker vi for eksempel en pil fra punktet $ A=(1, 1,\newline 0) $ til $B=(3, 4, 4) $, vil $ \vec{u} $ fra \fref{vekt1} dukke opp igjen, men denne gangen forskjøvet fra grunnstillingen.
\fig{vekt2}{Vektoren $ \vec{u} $ to forsjellige plasser i rommet\label{vecu}}
At vi ser noe grafisk er sjeldent et tilstrekkelig bevis, og skal vi sjekke om to vektorer er like bør dette gjøres ved regning. I vårt tilfelle ønsker vi å vite hvilken vektor som bringer oss fra $ A $ til $ B $, og da må vi finne differansen mellom koordinatene:
\[ [3-1, 4-1, 4-0]=[2, 3, 4] \]
Dermed har vi vist at nettopp nevnte $ \vec{u} $ er denne vektoren.\regv
\lvek
\lveke
\tssec{Regneregler for vektorer}
Av (\ref{vektmpunkt}) innser vi at hvis vektoren mellom $ A $ og $ B $ er betegnet som $ \vec{u} $, finner vi koordinatene til $ B $ ved å addere koordinatene til $ A $ med komponentene til $ \vec{u} $. Dette leder oss til den litt snodige konvensjonen at et punkt addert med en vektor blir et punkt\footnote{I mange matematiske tekster blir punkt og vektorer sett på som det samme.}.\regv
\rgnreg
\newpage
\paraeto
\tssec{Lengden av en vektor}\index{vektor!lengden av}
Vi har sett hvordan komponentene bestemmer hvor vi ender når vi følger en vektor, men de angir også den korteste avstanden mellom punktet vi starter fra og punktet vi ender i. For en vektor $ \vec{u} $ kaller vi denne asvstanden \textit{lengden} av $ \vec{u} $, som vi skriver som $ |\vec{u}| $. La oss prøve å finne lengden av en vektor $ {\vec{u}=[x_1, y_1, z_1]} $, som skissert i \fref{lenu}. Grafisk er lengden avstanden fra den butte enden til pilspissen.
\fig{vekt3}{\label{lenu}}
Av $ |\vec{u}| $ og lengdene $ z_1 $ og $ \hat{u}=\sqrt{x_1^2+y_1^2} $ danner vi en rettvinklet trekant. Av Pytagoras' setning har vi da at lengden av $ \vec{u} $ er gitt som
\alg{
	|\vec{u}|&=\sqrt{\hat{u}^2+z_1^2} \\
	&= \sqrt{x_1^2+y_1^2+z_1^2} 
}
\len
\lene
\eks[2]{
	Finn lengden av vektoren ${\vec{a}= [-9, 18, 27]} $.
	
	\sv
	Ved å bruke \eqref{fellesfakt} sparer vi oss for kvadrater av store tall:
	\[ [-9, 18, 27]=9[-1, 2, 3] \]
	Lengden blir da (se opg. \ref{lenfakt})
	\alg{
		|\vec{a}|&= 9\sqrt{(-1)^2+2^2+3^2} \\
		&= 9\sqrt{14}
	}\vds
}
\section{Skalarproduktet}\index{skalarprodukt}
Vi skal nå se på definisjonen av en regneoperasjon mellom vektorer som kalles \textit{skalarpruduktet} (også kalt \textit{prikkproduktet} eller \textit{indreproduktet}). Skalarproduktet skiller seg ut ifra mange andre operasjoner fordi det kan regnes ut på to vidt forskjellige måter.\vsk

Ved den ene utregningen skal vi anvende vinkelen \textit{utspent} av to vektorer, som er den minste vinkelen\index{utspent!vinkel} mellom to vektorer med samme utgangspunkt. For to vektorer $ \vec{u} $ og $ \vec{v} $ skriver vi denne som $ \angle(\vec{u}, \vec{v}) $.
\begin{figure}
		\centering
	\begin{tikzpicture}[scale=2.5]
	\draw[-triangle 45,color=black] (0,0) -- (30:1.3) node[midway, anchor = north west] { $ \vec{v}$}; 
	\draw[-triangle 45,color=black] (0,0)--(160:1.3) node[midway, anchor=north east]{$\vec{u} $};
	\draw ([shift=(30:0.1)]0,0) arc (30:160:0.1);	
	\draw (95:0.2) node[]{$ \angle(\vec{u}, \vec{v})  $} ;	
	\end{tikzpicture} 
	\caption{$\angle(\vec{u}, \vec{v}) $ er vinkelen utspent av $ \vec{u} $ og $ \vec{v} $.}
\end{figure}
I vektorregning er det vanlig å oppgi vinkelmål i grader, for minst mulige vinkler betyr dette vinkler på intervallet $ [0^\circ, 180^\circ] $.
\tssec{Første definisjon}
Fra (\ref{lengd}) vet vi hvordan å finne lengden av en vektor. Si nå at vi har vektoren $ \vec{u}-\vec{v} $, hvor 
$ \vec{u}=[x_1, y_1, z_1] $ og $ \vec{v}=[x_2, y_2, z_2] $. Da er
\[ \vec{u}-\vec{v}=[x_1-x_2, y_1-y_2, z_1-z_2] \]
\newpage
Lengden kan dermed skrives som
\begin{align}
|\vec{u}-\vec{v}|&= \sqrt{(x_1-x_2)^2 + (y_1-y_2)^2 + (z_1-z_2)^2} \nonumber \\
&= \sqrt{x_1^2 - 2x_1 x_2 + x_2^2 + y_1^2 - 2y_1 y_2 + y_2^2 +z_1^2- 2z_1 z_2 + z_2^2 }\label{preskal}
\end{align}
Uttrykket på høyre side i ligningen over er skremmende langt. For å komprimere dette og lignende uttrykk, innfører vi skalarproduktet:\regv
\skalen
\skalene
\tssec{Andre definisjon}
Med den første definisjonen av skalarproduktet kan vi omskrive (\ref{preskal}) til
\begin{equation}
|\vec{u}-\vec{v}|= \sqrt{\vec{u}^{\,2} - 2\vec{u}\cdot\vec{v} + \vec{v}^{\,2}} \label{skl1}
\end{equation}
Videre merker vi oss følgende figur:
\begin{figure}
	\centering
\begin{tikzpicture}[scale=2.5]
\draw[-triangle 45,color=black] (0,0) -- (30:2) node[midway, anchor = north west] { $ \vec{u}$}; 
\draw[-triangle 45,color=black] (0,0)--(160:2) node[midway, anchor=north east]{$\vec{v} $};
\draw[-triangle 45,color=black] (160:2)--(30:2) node[midway, anchor=south]{$\vec{u}-\vec{v} $};	
\draw ([shift=(30:0.1)]0,0) arc (30:160:0.1);	
\draw (95:0.2) node[]{$ \theta $} ;	
\end{tikzpicture} 
\caption{$ \theta= \angle(\vec{u}, \vec{v}) $.}
\end{figure}

Av cosinussetningen\footnote{For en trekant med sider $ a $, $ b $ og $ c $, hvor $ a $ og $ b $ utspenner vinkelen $ v $, har vi at\[ c^2 = a^2+b^2-2ab\cos v \]} og (\ref{skl1}) har vi at
\alg{
|(\vec{v}-\vec{u})|^2&= |\vec{v}|^2+|\vec{u}|^2-2\vec{u}||\vec{v}|\cos \theta \\
\vec{v}^{\,2}- 2\vec{u}\cdot \vec{v} + \vec{u}^{\,2} &= \vec{v}^{\,2}+\vec{u}^{\,2}-2|\vec{u}||\vec{v}|\cos \theta \\
\vec{u}\cdot \vec{v} &= |\vec{u}||\vec{v}|\cos \theta
}
\skalto
\skaltoe
\skaltoeto 
\tssec{Regneregler}
Den observante leser har allerede lagt merke til at skalarproduktet av vektorer har mye til felles med pruktet av to tall. Dette blir enda mer tydelig når man ser på flere regneoperasjoner mellom vektorer:\regv
\skalrgn
\eks{Forkort uttrykket
	\[   \vec{b}\cdot(\vec{a}+\vec{c}) + \vec{a}\cdot(\vec{a}+\vec{b})+\vec{b}^{\,2} \]
	
	når du vet at $ \vec{b}\cdot\vec{c}=0 $. \\
	
	\sv
	\algv{\vec{b}\cdot(\vec{a}+\vec{c}) + \vec{a}\cdot(\vec{a}+\vec{b})+\vec{b}^{\,2} &= \vec{b}\cdot\vec{a}+\vec{b}\cdot\vec{c} + \vec{a}\cdot\vec{a}+\vec{a}\cdot\vec{b}+\vec{b}^2 \\
		&= \vec{a}^{\,2} + 2 \vec{a}\cdot \vec{b} + \vec{b}^{\,2} \\
		&= \left(\vec{a}+ \vec{b}\right)^2
	}\vds
}
\section{Vinkelrette og parallelle vektorer}
\tssec{Vinkelrette vektorer}
Fra (\ref{skal2}) kan vi gjøre en viktig observasjon. Hvis $ \vec{u} $ og $ \vec{v} $ står vinkelrett\footnote{''Vinkelrett på'' omtales gjerne som ''\textit{normalt} på''.} på hverandre, utspenner de vinkelen $90^\circ $. I så fall er $ {\cos \theta=0} $, og da blir\vs
\alg{
\vec{u}\cdot\vec{v}&=|\vec{u}| |\vec{v}|  \cos \theta	\\
&= 0
	}
Kan vi få ${ \vec{u}\cdot\vec{v} =0}$ om vinkelen mellom $ \vec{u} $ og $ \vec{v} $ \textit{ikke} er $ 90^\circ $? Fordi $ \theta\in[0^\circ, 180^\circ] $, er det bare $ {\theta=90^\circ }$ som gir at $ {\cos \theta=0 }$. Skal skalarproduktet bli 0 for andre vinkler, må derfor lengden av $ \vec{u} $ eller $ \vec{v} $ være 0. Den eneste vektoren med denne lengden er \textit{nullvektoren} ${\vec{0}=[0, 0, 0] }$, som rett og slett ikke har noen retning\footnote{Eventuelt kan man hevde at den peker i alle retninger! Forøvrig spiller nullvektoren en minimal rolle i R2-faget.}. Det er likevel vanlig å definere at nullvektoren står vinkelrett på \textit{alle} vektorer. \newpage
\vink
\eks[1]{
Sjekk om vektorene $ {\vec{a}=[5, -3, 2]} $ og $ {\vec{b}=[2, 4, 1]} $ er ortogonale.

\sv \vds
\algv{
\vec{a}\cdot\vec{b}&=[5, -3, 2]\cdot[2, 4, 1] \\
&=10-12+2 \\
&=0
}
Altså er $ \vec{a}\perp\vec{b} $.
}
\tssec{Parallelle vektorer}\vs
\begin{figure}
	\centering
	\begin{tikzpicture}[scale=1]
	\draw[-triangle 45,color=black] (0,0) -- (1,1) node[midway, anchor = north west] { $ \vec{u}$}; 
	\draw[-triangle 45,color=black] (3,0)--(5, 2) node[midway, anchor=north west]{$\vec{v}=2\vec{u} $};
	\end{tikzpicture} 
	\caption{$ \vec{u}||\vec{v} $}
\end{figure}
Hvis en vektor $ \vec{v} $ kan skrives som et multiplum\footnote{Hvis vi for to vektorer $ \vec{u} $ og $ \vec{v} $ og en konstant $ t $ kan skrive $ {\vec{v}=t\vec{u}} $, sier vi at $ \vec{v} $ er et multiplum av $ \vec{u} $.}
av en vektor $ \vec{u} $, er $ \vec{u}$ og $ \vec{v} $ parallelle. Vi lar figuren over stå som foreløpig forklaring for dette, og går videre til å finne en metode for å teste om to vektorer er parallelle.\vsk

Si at vi har vektorene $ {\vec{u}=[x_1, y_1, z_1]} $ og $ {\vec{v}=[x_2, y_2, z_2]} $. Tenk nå at forholdstallet mellom hver korresponderende komponent er det samme, altså et tall \textit{c}:
\[ \frac{x_2}{x_1}=\frac{y_2}{y_1}=\frac{z_2}{z_1}=c \] \newpage
Altså er $ {x_2=cx_1} $, $ {y_2=cy_1} $ og $ {z_2=cz_1} $. Dette betyr at $ \vec{v} $ kan skrives som\vs
\alg{
\vec{v}&=[cx_1, cy_1, cz_1] \\
&=c[x_1, y_1, z_1] \\
&= c\vec{u}
}
Og dermed er $ \vec{u} $ og $ \vec{v} $ parallelle. Dette skriver vi som \y{\vec{u}\parallel\vec{v}}. \vsk

Hvis forholdtstallet mellom de korresponderende koordinatene \textit{ikke} er det samme, kan vi ikke skrive den ene vektoren som et multiplum av den andre, da er $ \vec{u} $ og $ \vec{v} $ \textsl{ikke} parallelle. \regv
\para
\parae
\eks[2]{
Finn $ s $ og $ t $ slik at vektorene $ {\vec{u}=[-1, 2s, 4]} $ og $\vec{v}=[3, 18, \\4t+4]$ er parallelle.

\sv
Vi observerer at forholdet mellom $ x $-komponeten i $ \vec{v} $ og $ \vec{u} $ er $ {\frac{3}{-1}=-3} $. Hvis $ {\vec{u}\parallel\vec{v}} $, er altså $ {\vec{v}=-3\vec{u}} $. Vi kan derfor sette opp følgende ligning for $ s $:
\alg{
18 &= -3(2s) \\
s &= -3}
Videre må vi ha at
\algv{
4t+4 &= -3(4)\\
t &= -4
}\vds
}

\section{Determinanter} \index{determinant}
Vi skal her se på regneoperasjoner mellom vektorer som brukes for å finne \textit{determinanter}. Determinanter spiller en viktig rolle i den matematiske greinen som kalles \textit{lineær algebra}, men for vår del er de mer et middel som skal brukes for å finne \textit{vektorprodukt} i neste seksjon.\regv
\detto
\newpage
\eks{
Gitt vektorene $ {\vec{u}=[-1, 3] }$ og $ {\vec{v}=[-2, 4]} $. Bestem $ \det(\vec{u}, \vec{v})$.

\sv
	\vs	\algv{
	\det(\vec{u}, \vec{v}) &= \left|\begin{matrix}
		-1 & 3 \\
		-2 & 4
	\end{matrix}\right| \\
	&= (-1)4-3(-2)	\\
	&= 2
}\vds
}
\dettre
\dettree
\section{Vektorproduktet}\index{vektorprodukt}
\tssec{Definisjon}
Vi har sett hvordan vi ved skalarproduktet kan sjekke om to vektorer $ \vec{u} $ og $ \vec{v} $ står normalt på hverandre, men ofte kan vi isteden være interessert i å finne en vektor som står normalt på begge disse. En slik vektor får vi ved \textit{vektorproduktet} (også kalt \textit{kryssproduktet}\index{kryssprodukt}) av $ \vec{u} $ og $ \vec{v} $, som vi skriver som $ {\vec{u}\times\vec{v}} $.
\vsk

\textsl{Merk}: For skalarproduktet får vi en skalar (et tall), mens vi for vektorproduktet får en vektor. Det er derfor veldig viktig å skille symbolet $ \cdot $ fra $ \times $.\regv
\krypro
\newpage
\eks{
	Gitt vektorene $ {\vec{a}=[-3, 2, 3]} $ og $ {\vec{b}=[2, -2, 1]} $.\os
	\textbf{a)} Finn $ \vec{a}\times\vec{b} $.\os
	\textbf{b)} Vis at vektoren du fant i a) står normalt på både $ \vec{a} $ og $ \vec{b} $. 
	
	\sv
	\textbf{a)} Vi bruker uttrykket fra (\ref{vp1}),	og regner ut følgende $ {3\times3} $ determinant:
	\[ \vec{a}\times\vec{b} =\left|\begin{matrix}
	\vec{e}_x & \vec{e}_y & \vec{e}_z \\
	-3 & 2 & 3 \\
	2 & -2 & 1
	\end{matrix}\right|  \]
	Vi får da at (se gjerne tilbake til eksempelet på side \pageref{eks33})
	\small
	\alg{
		\vec{a}\times\vec{b}&=\vec{e}_x\left|\begin{matrix}
			2 & 3 \\
			-2 & -1
		\end{matrix}\right|-\vec{e}_y\left|\begin{matrix}
			-3 & 3 \\
			2 & 1
		\end{matrix}\right|+\vec{e}_z\left|\begin{matrix}
			-3 & 2 \\
			2 & -2
		\end{matrix}\right|  \\
		&= \vec{e}_x(2\cdot1-3\cdot(-2))-\vec{e}_y(-3\cdot1-3\cdot2)+\vec{e}_z(-3\cdot (-2)-2\cdot2) \\
		&= 8\vec{e}_x+9\vec{e}_y+2\vec{e}_z \\
		&= [8, 9, 2]	
	}
\textbf{b)} To vektorer står normalt på hverandre dersom skalarproduktet av dem er 0:\vs \alg{
&[8, 9, 2]\cdot[-3,2,3]=-24+18+6=0 \os
&[8, 9, 2]\cdot[2,-2,1]=16-18+2=0
}
\vds
}
\kryproregn
\tssec{Vektorprodukt som areal og volum}\index{vektorprodukt!som areal og volum}
En anvendelse av vektorproduktet (og skalarproduktet) er å finne arealet og volumet av noen geometriske former som kan sies å være \textsl{utspent}\index{utspent!geometrisk figur} av vektorer. Med dette mener vi at to eller tre vektorer som starter i samme utgangspunkt, utgjør grunnlaget for en \textit{trekant}, et \textit{parallellogram}, et \textit{parallellepiped}, en \textit{pyramide} eller et \textit{tetraeder}.
\newpage
\begin{figure}
	\centering
	\subfloat[Trekant]{\includegraphics[]{\asym{trk}}}\qquad
	\subfloat[Parallellogram]{\includegraphics[]{\asym{parll}}}\\
	\subfloat[Parallellepiped]{\includegraphics[]{\asym{par}}}
	\subfloat[Pyramide]{\includegraphics[]{\asym{pyram}}}\quad
	\subfloat[Tetraeder]{\includegraphics[]{\asym{tetra}}}
	\captionof{figure}{Geometriske former utspent av vektorene $ \vec{u} $, $ \vec{v} $ og $ \vec{w} $.}\index{parallellogram}\index{trekant}\index{parallellepiped}\index{pyramide}\index{tetreaeder}
\end{figure}

\kryarvo

\newpage
\tsec{Forklaringer}
\subsection*{Parallelle vektorer et multiplum av hverandre}
Vi skal nå vise hvorfor $ {\vec{u}=[x_1, y_1, z_1]} $ og $ {\vec{v}=[x_2, y_2, z_2] }$ må være et multiplum av hverandre for at de skal være parallelle. \vsk

Ligning (\ref{vekpar}) forteller oss at $ |\vec{u}\times\vec{v}| $ tilsvarer arealet av parallellogramet utspent av $ \vec{u} $ og $ \vec{v} $. Dette arealet kan bare ha verdien 0 hvis $ \vec{u} $ og $ \vec{v} $ er parallelle, og den eneste vektoren med lengde 0 er nullvektoren $ [0,0,0] $. Kombinerer vi dette kravet med (\ref{vekpro}), får vi at
\[ [y_1z_2-z_1y_2, -(x_1z_2-z_1x_2), x_1y_2-y_1x_2]=[0, 0, 0] \]
Uttrykket over gir oss tre ligninger
\alg{
y_1z_2-z_1y_2&= 0	 \os
x_1z_2-z_1x_2&= 0 \os
x_1y_2-y_1x_2&=0
	}
som vi kan omskrive til
\alg{
\frac{y_1}{y_2}&=\frac{z_1}{z_2} \br	
\frac{x_1}{x_2}&=\frac{z_1}{z_2} \br
\frac{x_1}{x_2}&=\frac{y_1}{y_2}
	}
Til slutt kan vi samle alle tre til én ligning:
\[\frac{x_1}{x_2}= \frac{y_1}{y_2}=\frac{z_1}{z_2} \]
Altså må forholdet mellom hver korresponderende komponent være likt. \vsk

Videre må det finnes tre tall $ a $, $ b $ og $ c $ slik at
\[ \vec{u}=[ax_2, by_2, cz_3] \]
Skal $ \vec{u} $ være parallell med $ \vec{v} $ har vi følgende krav:
\[ \frac{ax_2}{x_2}= \frac{by_2}{y_2}=\frac{cz_2}{z_2} \]
Forkorter vi disse brøkene, finner vi at \y{a=b=c }. Da er
\alg{
\vec{u}&= [ax_2, ay_2, az_2] \\
&= a[x_2, y_2, z_2]\\
&= a\vec{u}	
	}
Herav må $ \vec{u} $ og $ \vec{v} $ være et multiplum av hverandre skal vi ha at $ \vec{u}||\vec{v} $.
\subsection*{Vektorproduktet}
Hensikten med vektorproduktet er å innføre en regneoperasjon som gir oss en vektor $  {\vec{w}=[x, y, z]} $ som står normalt på to andre vektorer $ {\vec{u}=[a, b, c]} $ og $ {\vec{v}=[d, e, f]} $. For at dette skal være sant, vet vi av (\ref{vnkr}) at \vs
\begin{align}
	\vec{u}\cdot\vec{w}&=0 \nonumber \\ 
	ax+by+cz &=0 \nonumber\\
	ax+by &= -cz\label{vktp2}\\
	& \nonumber \\
	\vec{v}\cdot\vec{w}&=0 \nonumber \\ 
	dx+ey+fz&=0  \nonumber \\
	dx+ey&=-fz\label{vktp5}
\end{align}
Vi har altså to forskjellige ligninger som kan hjelpe oss med å finne de tre ukjente størrelsene $ x, y $ og $ z$. Dette kalles at man har en ligning med \textit{én fri variabel}. Hvis vi velger $ z $ som fri variabel betyr dette kort fortalt at vi kan finne et uttrykk for $ x $ og $ y $ som vil oppfylle (\ref{vktp2}) og (\ref{vktp5}) for et hvilket som helst valg av $ z $. \vsk

Vi starter med å finne et uttrykk for $ x $. Først multipliserer vi (\ref{vktp5}) med $ \frac{b}{e} $, og subtraherer deretter venstre- og høyresiden fra denne ligningen med henholdsvis venstre- og høyresiden fra ligning (\ref{vktp2}):
\alg{
	ax+by -\left(\frac{bdx}{e}+by\right)&= -cz-\left(-\frac{bfz}{e}\right) \\
	ax - \frac{bdx}{e}&=-cz-\left(-\frac{bfz}{e}\right)
	}
Hvis vi videre multipliserer med $ e $, og deretter antar at $ ae-bd\neq0 $, får vi at
\begin{align}
	aex -bdx &= bfz-cez \nonumber\\
	(ae-bd)x &= (bf-ce)z\nonumber \\
	x &= \frac{bf-ce}{ae-bd}z \label{x}
\end{align}
Med omtrent samme framgangsmåte og identisk antakelse finner vi et uttrykk for $ y $:
\begin{align}
	ax+by -\left(ax+\frac{aey}{d}\right)&= -cz-\left(-\frac{afz}{d}\right) \nonumber\\
	(bd-ae)y &= (af-cd)z \nonumber\\
	y &= \frac{af-cd}{bd-ae}z \label{y}
\end{align}
Som nevnt kan $ z $ velges fritt, og vi ser av (\ref{x}) og (\ref{y}) at valget $ {z=ae-bd} $ gir oss følgende fine uttrykk:
\alg{
	x &= bf-ce \\
	y &= -(af-cd) \\
	z &= ae-bd
	} 
Dette samsvarer med (\ref{vekpro}). \vsk

For å komme fram til likhetene over har vi antatt at $ {z=ae-bd\neq 0} $, men det er fristende å sjekke om uttrykkene vi nettopp har funnet oppfyller (\ref{vktp2}) og (\ref{vktp5}) også når ${ z=ae-bd=0} $:
\alg{ax+by &=0 \\
	a(bf-ce)+-b(af-cd) &= 0 \\
	-(ae-bd)c &= 0\\
	0 &= 0 \\
	& \\
	dx + ey &= 0 \\
	d(bf-ce)-e(af-cd)&=0 \\
	-(ae-db)f &= 0 \\
	0 &= 0 
	}
Med $ z $ som fri variabel er altså (\ref{vktp2}) og (\ref{vktp5})  oppfylt for alle ${ z=ae-bd} $, dermed har vi funnet et uttrykk som alltid vil gi oss en vektor $ \vec{w} $ som er ortogonal med både $ \vec{u} $ og $ \vec{v} $. \vsk

Så lenge man bruker uttrykkene fra \eqref{x} og \eqref{y}, vil $ \vec{w} $ være parallell med vektoren gitt ved (\ref{vekpro}), uansett valg av $ z $. I tillegg kan vi få uttrykket fra (\ref{vekpro}) også om vi velger $ x $ eller $ y $ som fri variabel (det får bli opp til leseren å konstatere disse to påstandene).\label{allekrypropar} Av dette kan vi konkludere med at alle vektorer som står ortogonalt på både $ \vec{u} $ og $ \vec{v} $ er parallelle med vektoren gitt ved (\ref{vekpro}).

\subsection*{Lengden av vektorproduktet \label{lengdvekforkl}}
For å komme fram til det vi ønsker, skal vi benytte oss av \textit{Lagranges identitet}\footnote{Den spesielt interesserte finner utledningen for identieten i \hrv{lagrange}}\index{Lagranges identitet}. Denne sier at vi for to vektorer
$ \vec{v} $ og $ \vec{u} $ har at
\[\phantom{testingcloservl} |\vec{v}\times\vec{u}|^2=|\vec{v}|^2|\vec{u}|^2-(\vec{v}\cdot\vec{u}\,)^2 \tag{Lagranges identitet}\]
Ved å anvende (\ref{skal2}) og \eqref{1} kan vi derfor skrive
\alg{
|\vec{v}\times\vec{u}|^2&=|\vec{v}|^2|\vec{u}|^2-(\vec{v}\cdot\vec{u})^2 	\\
|\vec{v}\times\vec{u}|^2&=|\vec{v}|^2|\vec{u}|^2-|\vec{v}|^2|\vec{u}|^2\cos^2 \theta \\
|\vec{v}\times\vec{u}|^2&= |\vec{v}|^2|\vec{u}|^2(1-\cos^2 \theta) \\
|\vec{v}\times\vec{u}| &= |\vec{v}||\vec{u}|\sin \theta 	
	}
\begin{comment}
svar på gruble
	\subsection*{\boldmath $ 2\times2 $ determinanten som areal}
	\subimport{fig/}{det}
	La oss ta utgangspunkt i et parallellogram utspent av to vektorer $ \vec{u}=[a, b] $ og $ \vec{v}=[c, d] $ med en vinkel $ \theta \leq 90^\circ $ mellom seg, som skissert i \fref{det}. Fra klassisk arealregning og trigonometri vet vi at arealet $ A $ av parallallogrammet er:
	\[ A = |\vec{v}||\vec{u}|\sin \theta  \]
	Ved hjelp av identieten fra ref?? kan vi omskrive dette til:
	\begin{equation}
	A = |\vec{v}||\vec{u}|\cos (90^\circ -\theta) \label{ardet}
	\end{equation}
	Vi kan altså tolke $ A $ som skalarproduktet av $ \vec{u} $ og en vektor $ \vec{w} $ med samme lengde som $ \vec{v} $, men en retning slik at $ \angle(\vec{u}, \vec{w})=(90^\circ-\theta) $. Dette medfører at $\angle (\vec{v},\vec{w} )=90^\circ $, og da blir $ \vec{w}\cdot\vec{v}=0 $. Derfor må vi ha at:
	\[ \vec{w}=[-d, c] \quad\vee\quad \vec{w}=[d, -c] \] 
	Bare én av disse gir oss den ønskede retningen som fører oss tilbake til (\ref{ardet}), men vi merker oss følgende:
	\[\cos \angle(\vec{u}, \vec{w}) = \frac{bc-ad}{|\vec{u}||\vec{w}|} \quad \vee \quad \cos \angle(\vec{u}, \vec{w}) = -\frac{bc-ad}{|\vec{u}||\vec{w}|}\]
	De to mulige cosinusverdiene har altså eksakt samme tallverdi. Tar vi nå absoluttverdien av $ \vec{u}\cdot\vec{w} $, kan vi derfor være sikre på at vi har det samme uttrykket som i (\ref{ardet}):
	\alg{
	A &= |\vec{v}||\vec{u}|\cos (90^\circ -\theta)\\
	&= |\vec{u}\cdot\vec{w}| \\
	&= |ad-bc|}
	For $ \theta>90^\circ $ kan vi gå fram på akkurat samme måten som vi nå har gjort, og vi regner oss dermed som ferdige med forklaringen.
\end{comment}

\subsection*{Vektorproduktet som areal}
\fig{parlb}{Parallellogram med grunnlinje $ |\vec{u}| $ og høyde $ |\vec{v}|\sin \theta $.}
Arealet av et paralellogram er gitt som grunnlinja ganger høyden. For et parallellogram utspent av vektorene $ \vec{u} $ og $ \vec{v} $, tilsvarer dette produktet $ |\vec{u}||\vec{v}|\sin \theta $, som er det samme som lengden $ |\vec{u}\times\vec{v}| $. Arealet av trekanten utspent av $ \vec u $ og $ \vec{v} $ er halvparten av arealet av parallellogrammet.
\newpage
\subsection*{Vektorproduktet som volum \label{vekprsomvolforkl}}
\fig{par}{\label{vopar3}}
Volumet $ V $ av et parallellepidet tilsvarer grunnflaten $ A $ ganger høyden $ h $:
\begin{equation}
V = Ah \label{vopar0}
\end{equation}
\subimport{fig/}{vopar}
I \fref{vopar3} er grunnflaten $ A $ utspent av vektorene $ \vec{v} $ og $ \vec{v} $, og vi vet fra (\ref{vekpar}) at
\begin{equation}
A = |\vec{u}\times\vec{v}|  \label{vopar1}
\end{equation}
La $ \theta $ være vinkelen mellom  $ \vec{u}\times\vec{v} $ og $ \vec{w} $. Hvis $ {90^\circ\geq\theta\geq0} $, får vi en figur som skissert i \fref[a]{vopar}. Da er høyden $ h $ gitt som
\[ h = |\vec{w}\,|\cos \theta \]
Er derimot $ {180^\circ\geq\theta>90^\circ} $, får vi en figur som skissert i \fref[b]{vopar}. Da er
\[ h = -|\vec{w}\,|\cos \theta \]
For alle $ \theta\in[0^\circ, 180^\circ] $ kan vi derfor skrive
\begin{equation}
h = \big| |\vec{w}\,|\cos \theta \big| \label{vopar2}
\end{equation}
Av (\ref{vopar0}), (\ref{vopar1}), (\ref{vopar2}) og definisjonen av skalarproduktet har vi derfor at\vs
\alg{
	\big|\vec{u}\times\vec{v}\cdot\vec{w}\,\big|&= \big||\vec{u}\times\vec{v}||\vec{v}\,|\cos \theta\big|\\
	&=Ah \\
	&= V	
}
Av klassisk geometri har vi videre at
\begin{itemize}
	\item volumet av pyramiden utspent av $ \vec{u} $ og $ \vec{v} $ er $ \frac{1}{3} $ av volumet av parallellepipedet.
	\item volumet av tetraedet utspent av $ \vec{u} $ og $ \vec{v} $ er $ \frac{1}{6} $ av volumet av parallellepipedet.
\end{itemize} 
\end{document}