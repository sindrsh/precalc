\documentclass[english, 11 pt]{report}
%\input{/home/sindre/akademiet/R/preamb}
%\input{/Users/sindre/Desktop/akademiet/R/preamb}
%\input{/Volumes/TR2/R/preamb}
\input{/media/sindre/TR2/R/preamb}

\begin{document}
\subsection{4.63}
\begin{figure}[H]
	\centering
	\includegraphics[scale=0.5]{fig0}
\end{figure}

Vi starter med å tenge parallellepidet. Det ene vi har fått vite er at: \[ AB=3 \]
I et parallellepidet er alle motstående sider paralelle og like lange. Derfor er: \[ AE=CG=5 \text{ og } AD=BC=3 \] 

Husk videre at en vektor har en retning og en lengde, men kan befinne seg overalt. Ut ifra figuren ser man for eksemel at $ \overrightarrow{AE}=\overrightarrow{CG},\,\overrightarrow{BC}=\overrightarrow{AD}\text{ osv}  $. La oss definere:
\begin{align*}
\overrightarrow{CG} &= \vec{k} \\
\overrightarrow{AB} &= \vec{a} \\
\overrightarrow{BC} &= \vec{b}
\end{align*}

Vi kan da skrive: 
\begin{align*}
\angle BAD &= \angle(\vec a, \vec b) = 90^{\mathrm{o}} \\
\angle BAE &= \angle(\vec a, \vec k) = 60^{\mathrm{o}} \\
\angle DAE &= \angle(\vec b, \vec k) = 60^{\mathrm{o}} \\
\end{align*}

\begin{figure}[H]
	\centering
	\includegraphics[scale=0.5]{fig1}
\end{figure}

\textbf{a)} I tillegg til de vi allerede har definert, kaller vi vektoren mellom \textit{A} og \textit{C} for $ \vec c $. Vinkelen utspent av $ \vec a $ og $ \vec c $ kaller vi for \textit{u}. \\

Ut ifra figuren kan vi finne at:
\begin{equation}
\vec c=\vec a + \vec b \label{vek1}
\end{equation}

Lengden til $ \vec c $ er gitt som $ \sqrt{\vec {c}^{\;2}} $. Vi starter med å regne ut uttrykket inni rottegnet:
\begin{align*}
	\vec {c}^{\;2} &= (\vec a + \vec b)^2 \\
	 &= \vec {a}^{\;2} +2\vec a \cdot \vec b + \vec {b}^{\;2}
\end{align*}
Siden vinkelen utspent av $ \vec a $ og $ \vec b $ er $ 90^\mathrm{o} $, så er $ \vec a \cdot \vec b=0 $. Videre vet vi at $ \lvert \vec a \rvert =3$ og at $ \lvert \vec b \rvert =4$. Generelt har vi for alle vektorer $ \vec{v} $ at $ \vec{v}^{\,2}=|\vec v|^2 $. Vi får dermed:
\begin{align*}
\vec {c}^{\;2} &= 3^2+4^2 \\
|\vec c\,|^2 &=25 \\
|\vec c\,| &= 5 
\end{align*}

Ut ifra definisjonen av skalarproduktet av to vektorer vet vi at:
\[ \vec{a}\cdot \vec c = \lvert \vec a \rvert \lvert \vec c\, \rvert \cos u \] 
Igjen setter vi inn uttrykket fra (\ref{vek1}):
\begin{align*}
\vec a \cdot \vec c &=\lvert \vec a \rvert \lvert \vec c \rvert \cos u \\
\vec a \cdot (\vec a + \vec b)  &= 3 \cdot 5 \cos u \\
\vec{a}^{\,2} + \vec a \cdot \vec b & =  15 \cos u \\
3^2 &= 15 \cos u \\
\frac{9}{15} &=\cos u
\end{align*}

Over har vi nok en gang utnyttet at $ \vec a \cdot \vec b = 0 $. \\

Når vi søker vinkelen til en vektor, bruker vi bare vinkler i intervallet $ v \in [0,180^\mathrm{o}] $. Dermed blir :
\[u  = \cos^{-1} \left(\frac{9}{15}\right) \approx 51.1^\mathrm{o}  \] \\

\textbf{b)} 
\begin{figure}[H]
	\centering
	\includegraphics[scale=0.5]{fig2}
\end{figure}
Vi observerer at:
\begin{equation}
\vec{m} = \vec{a} + \vec b + \vec k \label{vek2}
\end{equation}

\begin{align*}
\vec{m}^{\,2} &= \left(\vec{a} + \vec b + \vec k\right)^2  \\
&= \vec{a}^{\,2} + \vec{b}^{\,2} + \vec{k}^{\,2} + 2\left( \vec a \cdot \vec b+ \vec a \cdot\vec k +\vec k \cdot \vec b  \right) \\
&= 3^2 + 4^2 + 5^2+ 2 \left(|\vec a| |\vec k| \cos 60 +|\vec k| |\vec b| \cos 60\right)  \\
& = 50 + 2 \left(3\cdot 5 \cdot \frac{1}{2} + 4\cdot 5 \cdot \frac{1}{2}\right) \\
& = 85
\end{align*}
Altså er $ |\vec m|=\sqrt{85} $. \\

\textbf{c)}

\begin{figure}[H]
	\centering
	\includegraphics[scale=0.5]{fig4}
\end{figure}
Av definisjonen for skalarproduktet skal vi ha at:
\[ \vec c \cdot \vec m = |\vec c| \cdot |\vec m| \cos \beta \]

Vi erstatter så $ \vec m $ med uttrykket fra ligning (2):
\begin{align*}
\vec c \cdot \vec m &= |\vec c| \cdot |\vec m| \cos \beta \\
\left(\vec a + \vec b + \vec k\right)\cdot \left(\vec a + \vec b\right) &= 5 \sqrt{85} \cos \beta \\
\vec{a}^{\,2}+ \vec a \cdot \vec k + \vec{b}^{\,2} + \vec b \cdot \vec k &= 5 \sqrt{85} \cos \beta \\
3^2 + 7.5 + 4^2 + 10 &= 5 \sqrt{85} \cos \beta \\
\frac{8.5}{\sqrt{85}} = \cos \beta
\end{align*}
\begin{align*}
\beta & = \cos^{-1}\left(\frac{8.5}{\sqrt{85}}\right) \\
 & \approx 22.8^{\mathrm{o}}
\end{align*} \\

\textbf{d)}

\begin{figure}[H]
	\centering
	\includegraphics[scale=0.5]{fig3}
\end{figure}

Når vi står i punkt \textit{C} kan vi komme til punkt \textit{E} ved å gå følgende rute:
\begin{equation}
\overrightarrow{CE}= \vec{i}= -\vec a -\vec b + \vec k \label{vek3}
\end{equation}
For å komme fra punkt \textit{C} til punkt \textit{M} kan man gå denne ruten:
\begin{align*}
\overrightarrow{CM} &= -\vec a -\vec b + \frac{1}{2} \vec m \\
& = -\vec a -\vec b + \frac{1}{2} \left(\vec a \vec b + \vec k\right) \\
&= - \frac{1}{2} \vec a - \frac{1}{2} \vec b + \frac{1}{2} \vec a \\
& = \frac{1}{2}\, \vec{i}
\end{align*}

Og dermed har vi vist at \textit{M} ligger på linja til \textit{CE}. \\

\textbf{e)} Vi krever at: 
\[  \vec i \cdot \vec m=0 \]
Ligningen over omskriver vi ved hjelp av ligning (\ref{vek2}) og (\ref{vek2}):
\begin{align*}
\overrightarrow{i}\cdot \vec m &= \left(-\vec a -\vec b + \vec k\right)\cdot   \left(\vec{a} + \vec b + \vec k\right) \\
&= -\vec{a}^{\,2} -\vec{b}^{\,2} +\vec{k}^{\,2} \\
& = -3^2-4^2 + 5^2 \\
& = 0
\end{align*}

Og dermed er kravet oppfylt.
\end{document}