\newcommand\rad{
	\rg[Relasjonen mellom grader og radianer]{\vspace{-5 pt}
		\begin{equation}\label{gradertilrad}
		1^\circ\ = \frac{\pi}{180}
		\end{equation}	\vs
	}	
}
\newcommand{\sincos}{\rg[Sinus og cosinus]{
	La enhetssirkelen være tegnet inn i et koordinatsystem med sentrum i origo, som vist i figuren under. 
	%\subimport{fig/}{korda}
	\begin{figure}[H]
	\centering
	\includegraphics{\asym{sincos}}
	\end{figure}
	La videre $ x $ representere en buelengde vandret i positiv (mot klokka) eller negativ retning fra punktet $ (1, 0) $ til et punkt $ (a_0, b_0) $. Da er\vs
\begin{align}
	\cos x &= a_0 \label{sindef}\\
	\sin x &= b_0\label{cosdef}
\end{align}\vds
	}}
\newcommand{\tang}{\rg[Tangens]{\vs	\vspace{-5pt}
	\begin{equation}\label{tandef}
	\tan x = \frac{\sin x}{\cos x} 
	\end{equation}\vs}}
\newcommand{\arcus}{\rg[Arcusuttrykkene]{
		Uttrykket
		\nreq{\text{atri}\;x = d}
		hvor tri erstattes med $ \sin, \cos $ eller $ \tan $, betyr at
		\nreq{\text{tri}\;d=x}
		\vds
	}
}
\newcommand{\arcuse}{	\eks[]{\vs
		\algv{
	\asin \left(\frac{1}{2}\right)&=\frac{\pi}{6}\br
	\sin\left(\frac{\pi}{6}\right)&= \frac{1}{2}	
	}	\vs
	}}
\newcommand{\trien}{\rg[Trigonometriske identiteter]{ 
		\vds
		\begin{align} 
		\cos (u+v) &=\cos u \cos v - \sin u \sin v \label{cuv} \\
		\cos (u-v) &=\cos u \cos v + \sin u \sin v \label{cu-v} \\
		\sin (u+v) &= \sin u \cos v +\cos u \sin v \label{suv} \\
		\sin (u-v) &= \sin u \cos v - \cos u \sin v \\
		\cos(-x) &= \cos x \label{-cosxcosx}\\
		\sin(-x) &= -\sin x \\
		\cos(x\pm\pi)&=-\cos x \\
		\sin(x\pm\pi)&=-\sin x \\
		\sin (2x) &= 2\cos x \sin x \label{sin2x}\\			
		\cos^2 x + \sin^2 x &= 1 \label{1}\br
		\cos\left(u-\frac{\pi}{2}\right)&=\sin u \label{cossomsi}\br
		\sin\left(u+\frac{\pi}{2}\right)&=\cos u \label{sinsomcos}		
		\end{align}
		\vs
		}}
\newcommand{\triene}{\eks[1]{
	Bruk de trigonometrisk identitetene til å finne eksaktverdien til $ \cos \left(\frac{\pi}{3}\right) $ og $ \sin \left(\frac{\pi}{3}\right) $ når du vet at $ {\sin\left(\frac{\pi}{6}\right)=\frac{1}{2}} $ og $ {\cos \left(\frac{\pi}{6}\right)=\frac{\sqrt{3}}{2}} $.\\
		
	\sv
	Vi har at
	\alg{
	\cos\left(\frac{\pi}{3}\right)&=\cos\left(\frac{\pi}{6}+\frac{\pi}{6}\right) \\[5 pt]
	&= \cos \left(\frac{\pi}{6}\right)\cos\left(\frac{\pi}{6}\right) - \sin \left(\frac{\pi}{6}\right)\sin\left(\frac{\pi}{6}\right) \\[5 pt]
	&= \frac{3}{4} -\frac{1}{4} \\[5 pt]
	&= \frac{1}{2}
		}
	og videre at
	\alg{
	\sin\left(\frac{\pi}{3}\right)&=\sin\left(\frac{\pi}{6}+\frac{\pi}{6}\right) \br
	&= \sin \left(\frac{\pi}{6}\right)\cos\left(\frac{\pi}{6}\right) + \cos \left(\frac{\pi}{6}\right)\sin\left(\frac{\pi}{6}\right) \br
	&= 2\left(\frac{\sqrt{3}}{4}\right) \br
	&= \frac{\sqrt{3}}{2}
		}\vs
	}}
\newcommand{\kvad}{\rg[Kvadratiske ligninger II]{For å løse ligninger på formen
	\begin{equation}
	a \cos^2 (kx) + b \sin^2 (kx) = d 
	\end{equation}
utnytter vi at $ {d=d(\cos^2 (kx) + \sin^2 (kx))} $. Vi dividerer så ligningen med $ \cos^2 (kx) $, og løser den resulterende tangensligningen.
	}}
\newcommand{\coslig}{\rg[Cosinusligninger ]{
		Gitt ligningen
		\begin{equation}
		\cos x=d \label{cosligorig} 
		\end{equation}
		For ${n\in \mathbb{Z}} $ har vi at
\begin{itemize}
	\item Hvis ${d\in (-1, 1)}$, har \eqref{cosligorig} løsningene
	\nreq{x=\pm\acos d+ 2\pi n \label{coslig}}
	\item Hvis $ {d=1}$, har \eqref{cosligorig} løsningene
	\nreq{ x=2\pi n \label{coslig1}}
	\item Hvis $ {d=-1}$, har \eqref{cosligorig}  løsningene
	\nreq{x=\pi+2\pi n \label{coslig_1}}
\end{itemize}\vs
	}
}
\newcommand{\coslige}{	\eks[1]{
		Løs ligningen:
		\[ 4\cos x = 2\sqrt{3} \] \vs
		\sv 
		Vi starter med å isolere cosinusuttrykket:
		\alg{
			4\cos x &= 2\sqrt{3} \\
			\cos x &= \frac{\sqrt{3}}{2}
		}
		Siden $ \acos\left(\frac{\sqrt{3}}{2}\right)=\frac{\pi}{6} $, har vi at
		\[ x =\pm \frac{\pi}{6}+2\pi n \]\vsb
	}
	}
\newcommand{\cosligeto}{\eks[2]{
		Finn løsningene til ligningen
		\[ 4\cos \left(\frac{\pi}{6} x-\frac{\pi}{3}\right) = 2\sqrt{3} \qquad,\qquad x\in [-9,4]\] \vs
		\sv
		Fra svaret i \textit{Eksempel 1} på forrige side vet vi at kjernen må oppfylle kravet
		\[ \frac{\pi}{6} x-\frac{\pi}{3} = \pm \frac{\pi}{6}+2\pi n \]
Vi må altså enten ha at
		\alg{
	\frac{\pi}{6} x &= \frac{\pi}{6}+\frac{\pi}{3}+2\pi n \br
	\frac{\pi}{6} &= \frac{\pi}{2}+2\pi n \br
	x &= 3+12n
}
eller at\vspace{-8 pt}
		\alg{
	\frac{\pi}{6} x &= \frac{\pi}{3}-\frac{\pi}{6}+2\pi n \br
	\frac{\pi}{6} x &= \frac{\pi}{6}+2\pi n \br
	x &= 1+12n
}
På intervallet $ [-9, 4] $ vil ${ x\in \{-9, 1, 3\}}$ oppfylle dette kravet.
	}}
\newcommand{\sinlig}{\rg[Sinusligninger]{
		
		Gitt ligningen 
		\begin{equation}
		\sin x=d\ \label{silig} 
		\end{equation}
		For ${n\in \mathbb{Z}} $ har vi at
		\begin{itemize}
		\item 	Hvis ${d\in (-1, 1)}$ har \eqref{silig} løsningene
			\nreq{
		x=\asin d+ 2\pi n\quad\vee\quad x=\pi-\asin d + 2\pi n \label{sinlig}	
	}		
		\item 	Hvis $ {d=1}$ har (\ref{silig}) løsningene
			\nreq{
				x=\frac{\pi}{2}+2\pi n	\label{sinlig1}
			}
		\item 	Hvis $ {d=-1}$ har (\ref{silig}) løsningene
			\nreq{
				x=-\frac{\pi}{2}+2\pi n	\label{sinlig_1}
			}\vs \vs
		\end{itemize}
	}
}
\newcommand{\sinlige}{	\eks{
		Løs ligningen
		\[ 2 \sin x = \sqrt{2} \]\vs
		\sv
		Vi kan skrive
		\algv{
			\sin x = \frac{\sqrt{2}}{2}
		}
		$ \asin\left(\frac{\sqrt{2}}{2}\right)=\frac{\pi}{4} $, da er $ x $ enten gitt som
		\[ x = \frac{\pi}{4}+2\pi n \]
		eller som
		\algv{
			x &= \pi-\frac{\pi}{4} + 2\pi n \\
			&= \frac{3\pi}{4} + 2\pi n
		}
	\textsl{Merk}: Det kan være praktisk å bli fortrolig med sinus- og cosinusverdiene til tallene i \fref{enh}. Da vil man direkte se at $ \frac{\pi}{4} $ og $ \frac{3\pi}{4} $ er tall med samme sinusverdi, men at de ligger i hver sin kvadrant. Legges $ 2\pi n$ til hver av dem, har man funnet alle løsninger. Man unngår da å regne ut $ \pi $ minus et tall, dette er tidsbesparende og minsker i tillegg sjansen for regnefeil.
	}}
\newcommand{\tanlig}{\rg[Tangensligninger ]{
		Ligningen \begin{equation}
		\tan x=d\  \label{tanlig}
		\end{equation}har løsningene		
		\nreq{x=\atan d + \pi n \label{tansols}}
		hvor $ {d\in\mathbb{R}} $ og ${n\in \mathbb{Z}} $.}
}
\newcommand{\tanlige}{	\eks{
		Løs ligningen
		\[ \sqrt{3}\tan (2x) = 1 \]\vs
		\sv
		Vi starter med å isolere tangensuttrykket:
		\alg{
			\sqrt{3}\tan (2x) &= 1 \\
			\tan (2x) &= \frac{1}{\sqrt{3}}	
		}
		Siden $ \atan \left(\frac{1}{\sqrt{3}}\right) = \frac{\pi}{6} $, får vi at
		\alg{
			2x &= \frac{\pi}{6}+\pi n \br
			x &= \frac{1}{2}\left(\frac{\pi}{6} + \pi n\right)	
		}\vs
	}}	
\newcommand{\kvaden}{\rg[Kvadratiske ligninger I]{
		Når vi skal løse ligninger av typen 
		\begin{equation}
		 a\,\text{tri}^2\,x+b\,\text{tri}\,x+c = 0 
		\end{equation}
		hvor $ a $, $ b $ og $ c $ er konstanter og tri erstattes med sin, cos eller tan, gjør vi følgende:
		\begin{enumerate}
			\item løser ligningen mhp. $\text{tri}\, x $
			\item løser de nye ligningene mhp. $ x $
		\end{enumerate}\vs
	}
	}
\newcommand{\kvadene}{\eks[1]{
		Løs ligingen
		\[ \cos^2 x -3 \cos x - 4=0 \]\vs
		\sv
		Vi starter med å løse andregradsligningen mhp. $ \cos x $. Da $ 1(-4)=-4 $ og $ 1-4=-3 $, får vi at (se \hrv{abc})
		\[ \cos x = -1 \quad \vee \quad \cos x = 4 \]
		Siden $ {\cos x=4} $ ikke har noen reell løsning, trenger vi bare å løse ligningen $ {\cos x=-1 }$. Vi får da at
		\[ x = \pi + 2\pi n \]\vds
	}}
%Funksjoner		
\newcommand{\sinf}{\rg[Sinusfunksjonen]{
		En funksjon $ f(x) $ på formen
		\begin{equation}\label{sinfunk}
		f(x)= a \sin (kx+c) + d
		\end{equation}
		kalles en sinusfunksjon med amplitude $ |a| $, bølgetall $ k $, fase $ c $ og likevektslinje $ y=d $. \vsk
		
		$ c $ er gitt ved ligningen
		\begin{equation}\label{sinc}
		k x_1+ c = \frac{\pi}{2}
		\end{equation}
		hvor $ x_1 $ er $ x $-verdien til et toppunkt. \vsk
		
		Funksjonen har ekstremalpunkter for alle $ x $ der
		\begin{equation}\label{sinmax}
		kx +c = \pm \frac{\pi}{2}+2\pi n
		\end{equation}
		for alle $ n\in \mathbb{Z} $.
	}}
\newcommand{\rel}{\rg[Relasjonene mellom sinus- og cosinusfunksjoner]{
		\vspace{-11 pt}
		\begin{align}
		&\cos \left(kx+c -\frac{\pi}{2}\right)= \sin(kx+c)	\\[5 pt]
		& \sin\left(kx +c + \frac{\pi}{2}\right)= \cos(kx+c)
		\end{align}
	}}
\newcommand{\komb}{\rg[Sinus og cosinus kombinert ]{
		Vi kan skrive
		\begin{equation}\label{rsin}
		a \cos (kx) +b \sin (kx)= r\sin(kx+c)
		\end{equation}
		der $ r=\sqrt{a^2+b^2} $ og hvor
		\begin{align}
				\cos c &= \frac{b}{r}\\[5 pt]
			\sin c &= \frac{a}{r}
		\end{align}
		\vs}}	
\newcommand{\kombe}{	
	\eks{
		Skriv om $ \sqrt{3} \sin(\pi x)-\cos(\pi x) $ til et sinusuttrykk. \\
		
		\sv
		Vi starter med å finne $ r $:
		\begin{align*}
		r&=\sqrt{\sqrt{3}^{\,2}+(-1)^2}  \\
		&= \sqrt{4} \\
		&= 2
		\end{align*}
		Videre krever vi at
		\algv{
			\cos c &= \frac{\sqrt{3}}{2}\\[5 pt]
			\sin c &= -\frac{1}{2}}
		Tallet $ {c=-\frac{\pi}{6}} $ oppfyller disse kravene, derfor er
		\[ \sqrt{3} \sin(\pi x)-\cos(\pi x) = 2 \sin\left(\pi x-\frac{\pi}{6}\right) \]	
		
		Se \hrv{lostriglig} for tips til hvordan å finne $ c $ når tallet ikke ligger i første kvadrant.		
	}}				
\newcommand{\asinbcos}{ 
	\rg[\boldmath $ a\sin (kx) + b\cos (kx) =0 $]{
	Ligningen
	\begin{equation}
	a\sin (kx) + b\cos (kx) =0 \label{asinbcos00}
	\end{equation}
	løses ved å dele begge sider med $ \cos (kx) $ og deretter løse den resulterende tangensligningen.
}}
\newcommand{\asinbcosd}{\rg[\boldmath $ a\sin (kx) + b\cos (kx) = d$]{
		Ligningen 
		\begin{equation}
		a\sin (kx) + b\cos (kx) = d 
		\end{equation}
		kan løses ved å omforme venstresiden til et reint sinusuttrykk, og deretter løse den resulterende sinusligningen.
}}

\newcommand{\tanf}{\rg[Tangensfunksjoner]{
		Funksjonen
		\begin{equation}\label{tanfunk}
		f(x)=a \tan(kx + c)+d
		\end{equation}
		har vertikale asymptoter for alle $ x $ der
		\begin{equation}\
		kx+c = \pm \frac{\pi}{2}+\pi n
		\end{equation}
		for $ n\in \mathbb{Z} $.\vsk\\
		
		Perioden $ P $ er gitt ved relasjonen
		\begin{equation}\
		P = \frac{\pi}{k}
		\end{equation}\vs
}}